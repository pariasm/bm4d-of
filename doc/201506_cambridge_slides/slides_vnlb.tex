% \documentclass{beamer}
\documentclass[mathserif, 8pt]{beamer}

% prevent 'no room for a new \count' or 
% 'no room for a new \dimem' errors
\usepackage{etex}
%\reserveinserts{28}

\mode<presentation> {
% \usetheme{Madrid}
%\usetheme{default}
\usetheme{Boadilla}
% \usecolortheme{beaver}
% \usecolortheme{rose}
}
% removes navigation symbols
\setbeamertemplate{navigation symbols}{}

\usepackage[english,activeacute]{babel}
\usepackage[english]{layout}
\usepackage{amsfonts,amsmath,amssymb,amsthm}
\usepackage{pgf}
%\usepackage{movie15}
\usepackage{graphicx}
\usepackage{rotating}
\usepackage{color}
\usepackage[utf8]{inputenc}
\usepackage[multidot]{grffile}
\usepackage{ucs}
\usepackage{dsfont}
% \usepackage[T1]{fontenc}
%\usepackage{subfigure}

% from rida's preamble
\usepackage{epsfig,xspace}
\usepackage{setspace}
\usepackage{threeparttable}
\usepackage{subfloat}
\usepackage{pstricks,pst-node}
\usepackage[normal,tight,center]{subfigure}
\usepackage{pifont}
\usepackage{multimedia}
%\usepackage{media9}
%\usepackage{enumitem}
\usepackage{animate}

\makeatletter
\let\zeropad\@anim@pad
\makeatother

\usepackage{caption}
\usepackage{multirow}

\usepackage{algorithm}
\usepackage{algorithmic}
\usepackage{setspace}

%\usepackage[compatible]{algpseudocode} % or \usepackage{algcompatible}
\renewcommand{\algorithmiccomment}[1]{\bgroup\hfill\small\textcolor{gray}{//~#1}\egroup}

% the following is used to bind frames with the same frame number
\usepackage{etoolbox}

\newcounter{multipleslide}

\makeatletter%
\newcommand{\multipleframe}{%
\setcounter{multipleslide}{\value{framenumber}}
\stepcounter{multipleslide}
\patchcmd{\beamer@@tmpl@footline}% <cmd>
	{\insertframenumber}% <search>
	{\themultipleslide}% <replace>
	{}% <success>
	{}% <failure>
}
\newcommand{\restoreframe}{%
\patchcmd{\beamer@@tmpl@footline}% <cmd>
	{\themultipleslide}% <search>
	{\insertframenumber}% <replace>
	{}% <success>
	{}% <failure>
\setcounter{framenumber}{\value{multipleslide}}%
}
\makeatother%





%%%%%%%%%%%%%%%%%%%%%%%%%%%%%%%%%%%%%%%
% Running 'pdflatex --shell-escape...' will automatically recognize
% the eps files in the includegraphics and convert them to pdf.
% No file rename nor change to the document is needed.
%%%%%%%%%%%%%%%%%%%%%%%%%%%%%%%%%%%%%%%
\usepackage{ifpdf}
\ifpdf
\DeclareGraphicsRule{.eps}{pdf}{.pdf}{`epstopdf #1}
\usepackage{epstopdf}
\epstopdfsetup{suffix=-\SourceExt-converted-to}
\pdfcompresslevel=0   %\pdfcompresslevel=9
\pdfcompresslevel0
\fi


% para poder incluir figuras .pdftex del xfig
\DeclareGraphicsRule{.pdftex}{pdf}{.pdftex}{}

\newcommand{\nada}[1]   {}

%\newcommand{\best}[1]{\textbf{\textcolor{MyOrange}{#1}}}
\newcommand{\best}[1]{#1}
\newcommand{\bsic}[1]{\textcolor{gray}{#1}}
\newcommand{\Bsic}[1]{\textcolor{gray}{\textbf{#1}}}
\newcommand{\Best}[1]{\textbf{\textcolor{MyOrangeBrighter}{#1}}}

\newcommand{\ma}[1]{\boldsymbol{#1}}
\newcommand{\ie}{\textit{i.e.} }
\newcommand{\eg}{\textit{e.g.} }
\newcommand{\etal}{\textit{et al}. }
\newcommand{\tras}[1]{#1^{\mathrm{T}}}
\newcommand{\herm}[1]{#1^{\mathrm{H}}}
\newcommand{\con}[1]{#1^{\mathrm{*}}}
\newcommand{\E}{\mathbb{E}}
\newcommand{\tech}[1]{\overline{#1}}
\newcommand{\nspace}{\!\!\!\!}
\newcommand{\nmbr}[1]{\oldstylenums{#1}}

\newcommand{\eps}{\varepsilon}
\newcommand{\R}{{\mathbb R}}
\newcommand{\Q}{{\mathbb Q}}
\newcommand{\N}{{\mathbb N}}
\newcommand{\Z}{{\mathbb Z}}
\newcommand{\C}{{\mathcal C}}
\renewcommand{\H}{{\mathcal H}}
\newcommand{\F}{{\mathcal F}}

\DeclareMathOperator*{\argmin}{arg\,min}
\DeclareMathOperator*{\argmax}{arg\,max}
\DeclareMathOperator*{\median}{median}



\definecolor{MyOrange}  {cmyk}{0,0.73,1.0,0}
\definecolor{MyOrangeBrighter}  {cmyk}{0,0.53,7.0,0}
\definecolor{MyGreen}   {cmyk}{1,0,1,0.0}

% Rida's colors and commands
\definecolor{Red}{rgb}{0.9,0.0,0.1}
\definecolor{Brown}{rgb}{0.55,0.27,0.1}
\definecolor{Brownie}{rgb}{0.75,0.27,0.1}
\definecolor{Yellow}{rgb}{1,1,0}
\definecolor{White}{rgb}{1,1,1}
\definecolor{formule}{rgb}{0.75,0.27,0.1}
\definecolor{formule}{rgb}{.7,1,0}
\definecolor{violet}{rgb}{0.7,0,.9}
\definecolor{darkgreen}{rgb}{0,.7,0}

%\newcommand{\reference}[1] {{\scriptsize \color{gray}  #1 }} 
\newcommand{\reference}[1] {{\color{gray} [#1]}} 
\newcommand{\rmk}[1]{{\color{Red} #1}}

%\theoremstyle{plain}\newtheorem{theorem}{Theorem}[chapter]
%\theoremstyle{plain}\newtheorem{proposition}{Proposition}[chapter]
%\theoremstyle{plain}\newtheorem{lemma}{Lemma}[chapter]
%\theoremstyle{definition}\newtheorem{definition}{Definition}[chapter]

% New definition of square root:
% it renames \sqrt as \oldsqrt
\let\oldsqrt\sqrt
% it defines the new \sqrt in terms of the old one
\def\sqrt{\mathpalette\DHLhksqrt}
\def\DHLhksqrt#1#2{%
\setbox0=\hbox{$#1\oldsqrt{#2\,}$}\dimen0=\ht0
\advance\dimen0-0.2\ht0
\setbox2=\hbox{\vrule height\ht0 depth -\dimen0}%
{\box0\lower0.4pt\box2}}


%gets rid of bottom navigation bars
% \setbeamertemplate{footline}[frame number]{}

%gets rid of navigation symbols
\setbeamertemplate{navigation symbols}{}

\title[Video non-local Bayes]{An empirical Bayesian method for video denoising}
\author[Pablo Arias]{Pablo Arias\\ Jean-Michel Morel}
\institute[CMLA - ENS Cachan]{{\small Centre de Math\'ematiques et de Leurs Applications (CMLA)\\ENS Cachan}}
\date[Challenges in Dynamical Imaging Data]{Challenges in Dynamical Imaging Data\\$\,$\\Turing Gateway for Mathematics\\Isaac Newton Institute, Cambridge, UK\\June 11, 2015}
%\date[DxO-CMLA report]{DxO-CMLA report\\$\,$\\June 9, 2015}
%\date[GTTI]{GTTI\\$\,$\\June 17, 2015}

%\author[Pablo Arias]{Pablo Arias$^*$, Vicent Caselles$^*$, Gabriele Facciolo$^\dagger$ \\ Rida Sadek$^*$, Guillermo Sapiro$^\ddagger$ \\ $\,$ \\  $\,$ \\ $\,$ \\ $\,$ \\
%\institute[UPF]{* Universitat Pompeu Fabra\\ $\dagger$ ENS Cachan \\ $\ddagger$ Duke University}

% \AtBeginSection[]  % "Beamer, do the following at the start of every section"
% {\begin{frame}<beamer>
% \frametitle{Outline} % make a frame titled "Outline"
% \tableofcontents[currentsection]  % show TOC and highlight current section
% \end{frame}}
%
% \AtBeginSubsection[]  % "Beamer, do the following at the start of every section"
% {\begin{frame}<beamer>
% \frametitle{Outline} % make a frame titled "Outline"
% \tableofcontents[currentsection,currentsubsection]  % show TOC and highlight current section
% \end{frame}}

%\graphicspath{{./figures/}}
\graphicspath{{./figures_local/}}

\begin{document}

\begin{frame}
    \titlepage
\end{frame}

\section{Introduction}

\begin{frame}{Introduction}

	\setbeamertemplate{itemize items}[triangle]

	Many applications use videos: video analysis, computer vision, video
	surveillance, movie postproduction\dots

	\bigskip

%	Digital videos have noise: 
%	\begin{itemize}
%		\item Quality of digital image sensors increases steadily, but so does
%			the number of pixels. 
%		\item Cheap cameras (e.g. mobile phones), are used
%			under challenging conditions (low light, high dynamic range, etc). 
%	\end{itemize}

	\bigskip

% 	Video denoising aims at pushing the limits on the conditions for acquiring
% 	high quality video, facilitating posterior applications (e.g.  video
% 	analysis, computer vision, video surveillance, movie postproduction). 

	\bigskip

	Video denoising is much less explored than image denoising.\\

	\bigskip

	\structure{Challenges in video denoising} are:
	\begin{itemize}
		\item Exploit \structure{efficiently} the additional source of redundancy given by the temporal consistency.
		\item Impose the same temporal consistency on the result. 
	\end{itemize}

	\bigskip

%	\begin{quote}
%		Artifacts introduced by the 
%		video denoising algorithm should be temporally consistent!
%	\end{quote}

\end{frame}

% \begin{frame}{Review of video denoising methods}
% 
% 	\setbeamertemplate{itemize items}[triangle]
% 
% %	\begin{itemize}\itemsep=.5cm
% %			\item%
% 			Spatio-temporal neighborhood filtering \reference{Brailean et al.'95}.
% 			\begin{itemize}\itemsep=.3cm
% 				\item Temporal adaptive filters that suppress filtering when motion is detected.
% 				\item Temporal compensated filters estimate motion trajectories and filter along.
% 			\end{itemize}
% 
% 			\pause
% 			\vspace{.5cm}
% 
% %		\item%
% 			Wavelet-based methods:
% 			\begin{itemize}\itemsep=.3cm
% 					\item Spatial wavelet shrinkage with temporal filtering of
% 						wavelet coefficients \reference{Yin et al.'06},
% 						\reference{Zlokolica et al.'06}.
% 					\item Spatial-temporal 3D wavelet shrinkage of full video \reference{Rajpoot et al.'03} \reference{Selesnick, Li'03}.
% 			\end{itemize}
% 
% 			\pause
% 			\vspace{.5cm}
% 
% %		\item%
% 			Non-local patch-based methods. See next slides\dots
% %			\begin{itemize}\itemsep=.3cm
% %				\item Non-local means without motion estimation
% %					\reference{Buades et al.'05}.
% %				\item V-BM3D extension of BM3D to video by searching for similar
% %					patches in a spatio-temporal neighborhood \reference{Dabov et al. '07}.
% %				\item K-SVD on 3D spatio-temporal patches \reference{Protter, Elad '07}.
% %				\item Non-local means of similar patches and their motion trajectories \reference{Liu, Freeman '10}.
% %				\item V-BM4D: collaborative filtering of 3D motion-compensated
% %					spatio-temporal patches \reference{Maggioni et al. '12}.
% %			\end{itemize}
% %	\end{itemize}
% 
% \end{frame}

%\begin{frame}{Review: patch-based video denoising methods}
%
%	\setbeamertemplate{itemize items}[triangle]
%
%	\begin{center}
%		\includegraphics<1>   [width=.7\textwidth]{figures/bus_slice100_s10.png}
%		\includegraphics<2-3> [width=.7\textwidth]{figures/video_slice_2d_nearest_neighbors.pdf}
%		\includegraphics<4>   [width=.7\textwidth]{figures/video_slice_3d_target_patch.pdf}
%		\includegraphics<5>   [width=.7\textwidth]{figures/video_slice_3dmc_target_patch.pdf}
%		\includegraphics<6>   [width=.7\textwidth]{figures/video_slice_3dmc_nearest_neighbors.pdf}
%		\includegraphics<7-8> [width=.7\textwidth]{figures/video_slice_3d_nearest_neighbors.pdf}
%	%	\includegraphics<1-10>[width=.7\textwidth]{figures/video_slice_2d_nearest_neighbors_s40.pdf}
%	%	\includegraphics<1-10>[width=.7\textwidth]{figures/video_slice_2d_target_patch.pdf}
%	%	\includegraphics<1-10>[width=.7\textwidth]{figures/video_slice_3dmc_search.pdf}
%	%	\includegraphics<4->[width=.7\textwidth]{figures/bus_slice100_s40.png}
%	%	\includegraphics<2>[width=.8\textwidth]{figures/bus_slice100_s40_vbm4d.png}
%	%	\includegraphics<3>[width=.8\textwidth]{figures/bus_slice100_s40_vnlb3d.png}
%	%	\includegraphics<2>[width=.7\textwidth]{figures/bus_slice100_s10.png}
%	%	\includegraphics<3>[width=.7\textwidth]{figures/bus_slice100_s10_vbm4d.png}
%	%	\includegraphics<4>[width=.7\textwidth]{figures/bus_slice100_s10_vnlb3d.png}
%	%	\includegraphics<5>[width=.7\textwidth]{figures/bus_slice100_s20.png}
%	%	\includegraphics<6>[width=.7\textwidth]{figures/bus_slice100_s20_vbm4d.png}
%	%	\includegraphics<7>[width=.7\textwidth]{figures/bus_slice100_s20_vnlb3d.png}
%
%		%Horizontal slice of \emph{bus} sequence at $y = 100$.\\AWGN
%		%of \only<1-3>{$\sigma=10$}\only<4->{$\sigma = 40$} added. Vertical dimension is time.
%		Horizontal slice of \emph{bus} sequence at $y = 100$.\\AWGN
%		of $\sigma=10$ added. Vertical dimension is time.
%	\end{center}
%
%	\pause
%
%	\begin{overlayarea}{\textwidth}{3cm}
%		\only<2-3>{
%			First approaches: filter similar \structure{2D patches in a
%			spatio-temporal search region} \reference{Buades et al.'05},
%			\reference{Dabov et al.'07}.
%
%			\vspace{.1cm}
%		}
%
%		\only<3-3>{
%			\begin{itemize}
%				\item<3-3> Improvement in PSNR: \structure{more} similar patches
%					and more \structure{similar}.
%				\item<3-3> No need to estimate motion.
%				\item<3-3> Flickery results for moderate/high noise: temporally inconsistent set of
%					similar patches.
%			\end{itemize}
%		}
%
%		\only<4-5>{
%			The alternative is to use \structure{spatio-temporal 3D patches}.
%
%			\vspace{.1cm}
%
%			\begin{itemize}
%				\item<4-5> Same filtering applied to each spatial slice of the 3D patch.
%				\item<4-5> %Increased temporal consistency of nearest neighbors: 
%					More reliable
%					patch distance by increased dimensionality of the patch\dots
%				\item<4-5> \dots without increasing the spatial size: small patches have more
%					matches.
%			\end{itemize}
%		}
%
%		\only<6>{
%			\reference{Liu, Freeman '10}, \reference{Maggioni et al.'12} 
%			motion-compensated 3D patches (patch trajectories).
%
%			\vspace{.1cm}
%
%			\begin{itemize}
%				\item<6> Temporal intra-patch filtering is easy since spatial slices are aligned.
%				\item<6> Only search patch trajectories starting in the same
%					frame.
%%				\item Patch trajectories %starting in the same frame 
%%					have the same spatial self-similarity pattern than 2D patches.
%				\item<6> The challenge is to realiably estimate the motion.%: motion is still
%		%			computed using 2D patches.
%			\end{itemize}
%
%			\vspace{.5cm}
%		}
%	
%		\only<7-8>{
%			What about non motion-compensated 3D patches?
%			\begin{itemize}
%				\item<7-8> No straightforward temporal intra-patch filtering.
%				\item<7-8> Need to search in a 3D domain to have temporal filtering.
%				\item<7-8> For approximately uniform motion, similar 2D patches will be 3D
%					similar.
%			\end{itemize}
%
%			\bigskip
%		}
%
%		\only<8>{\reference{Protter, Elad '07}: K-SVD on 3D spatio-temporal patches.}
%
%	\end{overlayarea}
%
%\end{frame}

\begin{frame}{Review: patch-based video denoising methods}

	\setbeamertemplate{itemize items}[triangle]

	\begin{center}
		\includegraphics<1>   [width=0.75\textwidth]{figures/bus_slice100_s10.png}
		\includegraphics<2-3> [width=0.75\textwidth]{figures/video_slice_2d_nearest_neighbors.pdf}
		\includegraphics<4>   [width=0.75\textwidth]{figures/video_slice_3d_target_patch.pdf}
		\includegraphics<5>   [width=0.75\textwidth]{figures/video_slice_3dmc_target_patch.pdf}
		\includegraphics<6>   [width=0.75\textwidth]{figures/video_slice_3dmc_nearest_neighbors.pdf}
		\includegraphics<7-8> [width=0.75\textwidth]{figures/video_slice_3d_nearest_neighbors.pdf}
	%	\includegraphics<1-10>[width=.7\textwidth]{figures/video_slice_2d_nearest_neighbors_s40.pdf}
	%	\includegraphics<1-10>[width=.7\textwidth]{figures/video_slice_2d_target_patch.pdf}
	%	\includegraphics<1-10>[width=.7\textwidth]{figures/video_slice_3dmc_search.pdf}
	%	\includegraphics<4->[width=.7\textwidth]{figures/bus_slice100_s40.png}
	%	\includegraphics<2>[width=.8\textwidth]{figures/bus_slice100_s40_vbm4d.png}
	%	\includegraphics<3>[width=.8\textwidth]{figures/bus_slice100_s40_vnlb3d.png}
	%	\includegraphics<2>[width=.7\textwidth]{figures/bus_slice100_s10.png}
	%	\includegraphics<3>[width=.7\textwidth]{figures/bus_slice100_s10_vbm4d.png}
	%	\includegraphics<4>[width=.7\textwidth]{figures/bus_slice100_s10_vnlb3d.png}
	%	\includegraphics<5>[width=.7\textwidth]{figures/bus_slice100_s20.png}
	%	\includegraphics<6>[width=.7\textwidth]{figures/bus_slice100_s20_vbm4d.png}
	%	\includegraphics<7>[width=.7\textwidth]{figures/bus_slice100_s20_vnlb3d.png}

		%Horizontal slice of \emph{bus} sequence at $y = 100$.\\AWGN
		%of \only<1-3>{$\sigma=10$}\only<4->{$\sigma = 40$} added. Vertical dimension is time.
		Horizontal slice of \emph{bus} sequence at $y = 100$.\\AWGN
		of $\sigma=10$ added. Vertical dimension is time.
	\end{center}

	\pause

	\bigskip

	\bigskip

	\begin{overlayarea}{\textwidth}{3cm}
		\only<2-3>{
			First approaches: filter similar \structure{2D patches in a
			spatio-temporal search region}.

			\vspace{.1cm}
		}

		\only<3-3>{
			\begin{itemize}\itemsep=.3cm
				\item<3-3> Non-local means \reference{Buades et al.'05}: average of similar 2D patches weighted by similarity.
				\item<3-3> V-BM3D \reference{Dabov et al.'07}: filtering of 3D stack of similar 2D patches in a transformed domain.
			\end{itemize}
		}

		\only<4-6>{
			Alternative is to use \structure{spatio-temporal 3D patches}.

			\vspace{.1cm}

			\begin{itemize}\itemsep=.3cm
				\item<6> Non-local means \reference{Liu, Freeman '10}: average of similar patches and their trajectories.
				\item<6> V-BM4D \reference{Maggioni et al.'12}: filtering of 4D stack of similar 3D motion-compensated patches in a transformed domain.
			\end{itemize}
		}

		\only<7-8>{
			What about non motion-compensated 3D patches?

			\vspace{.1cm}

			\begin{itemize}\itemsep=.3cm
				\item<7-8> \reference{Protter, Elad '07}: K-SVD on 3D spatio-temporal patches.
				\item<8> Video non-local Bayes \reference{this talk}.
			\end{itemize}
		}

	\end{overlayarea}

\end{frame}


\section{Bayesian video denoising}

\begin{frame}{A Gaussian linear model for patches}

	\structure{Observation model:} We assume additive white Gaussian noise (AGWN):
	\[v = u + n,\]
	where $u,v,n:\Omega\times \{1,\cdots,T\}\rightarrow\mathds R$ and $n(x,t)\sim \mathcal N(0,\sigma^2)$.

	\bigskip                        

	\pause

	\structure{\emph{A priori} model for 3D patches:} Let $\ma p$ be a
	$s_x\times s_x\times s_t$ patch from $u$. In a locality of $\ma p$, 
	the set of 3D patches from $u$ are assumed to follow a Gaussian distribution. 


%	\begin{equation*}
%		\mathds{P}(\ma p) = \mathcal N(\overline {\ma p}, C_{\ma p}) \propto \exp\left(-\frac12\langle \ma p - \overline{\ma p}, C_{\ma p}^{-1}(\ma p - \overline{\ma p})\rangle\right)
%	\end{equation*}

	\bigskip

	\pause

	\begin{block}{Local Gaussian model for 3D patches}
	Alltogether, we have the following local model for 3D patches:
	\begin{align*}
		\mathds{P}(\ma p) &= \mathcal N(\ma p\,|\,\overline {\ma p}, C) 
		                   \propto \exp\left(-\frac12\langle \ma p - \overline{\ma p}, 
		                                C^{-1}(\ma p - \overline{\ma p})\rangle\right)\\
		\mathds{P}(\ma q|\ma p) &= \mathcal N(\ma q\,|\,\ma p, \sigma^2 I) 
		             \propto \exp\left(-\frac1{2\sigma^2}\|\ma q - \ma p\|^2\right)
	\end{align*}
	Where 
%	\begin{equation*}
	$
		\quad \ma p  \,\, \text{is a patch from } u \text{ and }
		\quad \ma q  \,\, \text{is the corresponding patch from } v.
	$
%	\end{equation*}
	\end{block}

\end{frame}

\begin{frame}{Patch denoising via MAP inference}

	Assume for now that we know $\overline{\ma p}$ and $C$.

	\bigskip

	Given a noisy patch $\ma q$, we estimate its noiseless version 
	as the \emph{maximum a posteriori} (MAP):
	\begin{align*}
		\widetilde{\ma p} &:= \argmax_{\ma p} \mathds P(\ma p | \ma q) 
								 = \argmin_{\ma p} -\log \mathds P(\ma p | \ma q)\\
								&\,\,= \overline{\ma p} + C(C
								 + \sigma^2 I)^{-1}(\ma q - \overline{\ma p})
	\end{align*}

	\bigskip

	\pause

	If $C = U\Lambda U^T$ is the eigendecomposition of $C$, we have that
	\[U^T(\widetilde{\ma p} - \overline{\ma p}) \,\,\,\, = \,\,\,\, 
		\Lambda(\Lambda + \sigma^2 I)^{-1}\,\,\,\,
		U^T (\ma q - \overline{\ma p})\]

	\bigskip

	\pause

	\[W = \Lambda(\Lambda + \sigma^2I)^{-1}\quad\rightarrow \quad 
		w_{ii} = \frac{\lambda_i}{\lambda_i + \sigma^2}
	%	       = (1 + \sigma^2/\lambda_i)^{-1}
				 = (1 + \text{snr}_i^{-1})^{-1}\]

	\begin{center}
		\begin{tabular}[c]{c p{7cm} c}
		\structure{$\ma \Longrightarrow$} &
		\centering Diagonal operator in the basis of principal directions: \\ 
		\structure{Wiener filter on the principal components $\ma q$.} &
		\structure{$\ma \Longleftarrow$}
		\end{tabular}
	\end{center}

\end{frame}

\begin{frame}{Learning the local Gaussian models}

	Marginal distribution for noisy patches:
	$\mathds{P}(\ma q) = \mathcal{N}(\ma q \,|\,\overline{\ma p}, C + \sigma^2I).$

	\bigskip

	\bigskip

	\bigskip

	Given a set of patches $\ma q_i \sim \mathcal{N}(\ma q \,|\,\overline{\ma
	p}, C + \sigma^2I)$, $i = 1,\dots,n$, we have the following \emph{maximum likelihood} estimates
	of the parameters:

	\begin{equation*}
		\widehat{\overline{\ma p}} = \frac1n\sum_{i = 1}^n \ma q_i
		\quad\quad\text{and}\quad\quad\quad
		\widehat{C} = \frac1n\sum_{i = 1}^n \ma q_i^T\ma q_i - \sigma^2 I
	\end{equation*}

	\bigskip

	\bigskip

	\bigskip

	In practice, for a noisy patch $\ma q$, we look for its $n_{\text{sim}}$
	nearest neighbors and assume that they are samples from a single Gaussian
	model. 

\end{frame}

\begin{frame}{Additional assumption: low dimensional patch manifold}

	\structure{Assumption:} Locally, the set of noiseless patches
	can be approximated by a hyperplane of dimension $r$.

	\bigskip

	\structure{Linear Gaussian model:} This translates into a rank $r$
	covariance matrix: \[C = WW^T, \quad\quad\text{with }W \text{ a }d\times r \text{ full rank matrix}.\] 

	\bigskip

	\structure{ML learning:} The \emph{maximum likelihood} estimate of $C$ is then 
	\[\widehat{C}_r = \widehat{U}_r\widehat{\Lambda}_r\widehat{U}^T_r\]
	where $\widehat{U}_r$, $d\times r$ and $\widehat{\Lambda}_r =
	\text{diag}(\lambda_1, \lambda_2,\dots,\lambda_r)$ are the largest 
	$r$ eigenvectors and eigenvalues of $\widehat{C}$.

	\bigskip

	\structure{MAP inference:} Same as before, but can be computed more
	efficiently.


\end{frame}



\begin{frame}{Algorithm}

%	\caption{Video NL-Bayes}
%	\label{alg:nlbayes}

	\begin{columns}
	\begin{column}{0.8\textwidth}
	\begin{overprint}

		\onslide<1>
		\begin{algorithm}[H]
			\caption{Video NL-Bayes - Step 1: basic estimate}
			\begin{spacing}{1.3}
				\begin{algorithmic}[1]
				\REQUIRE Noisy video $v$, noise standard deviation
				$\sigma$\textcolor{white}{, basic estimate $u^{(1)}$}
				\ENSURE Basic estimate of noiseless video $\widetilde u^{(1)}$
				\STATE Set $\mathcal P = \{\ma q \,:\, \ma q \text{ patch of }  v\}$ \COMMENT{patches to process}
				\WHILE{$\mathcal P \neq \emptyset$}
				\STATE Get a patch $\ma q$ from $\mathcal P$ \COMMENT{``center'' of Gaussian model}
					\STATE Retrieve  the $n_{\text{sim}}$ nearest neighbors to $\ma
					q$ in a spatio-temporal volume around $\ma q$\textcolor{white}{. The
						distance is computed between the basic estimates
						$\widetilde{\ma p}^{(1)}$}
					\STATE Compute $\widehat{\overline{\ma p}}\,\!^{(1)}$ and
					$\widehat C_{r}^{(1)}$ \textcolor{white}{from the basic
						estimates $\widetilde{\ma p}_i^{(1)}$, \textbf{setting} $\ma{\sigma = 0}$.}
					\FORALL{$n_{\text{sim}}$ neighbors $\ma q_i$ of $\ma q$}
						\STATE Obtain the MAP estimate $\widetilde{\ma p}_i^{(1)}$
						\STATE Aggregate estimated patch on $\widetilde u^{(1)}$
						\STATE Remove $\ma q_i$ from $\mathcal P$
					\ENDFOR
				\ENDWHILE
	%			\FORALL{pixel $(x,t)$ in $\Omega\times {1,\dots,T}$}
	%			\STATE Obtain the basic estimate $\widetilde u^{(1)}(x,t)$ by averaging
	%			the values of all patches $\widetilde{\ma p}^{(1)}$ containing $(x,t)$.
	%			\ENDFOR
			\end{algorithmic}
			\end{spacing}
		\end{algorithm}

		\onslide<2>
		\begin{algorithm}[H]
		\caption{Video NL-Bayes - Step 2: final estimate}
			\begin{spacing}{1.3}
				\begin{algorithmic}[1]
				\REQUIRE Noisy video $v$, noise standard
				deviation $\sigma$, \structure{basic estimate $\widetilde u^{(1)}$} 
				\ENSURE Final estimate of noiseless video $\widetilde u^{(2)}$
				\STATE Set $\mathcal P = \{\ma q \,:\, \ma q \text{ patch of }  v\}$ \COMMENT{patches to process}
				\WHILE{$\mathcal P \neq \emptyset$}
				\STATE Get a patch $\ma q$ from $\mathcal P$ \COMMENT{``center'' of Gaussian model}
					\STATE Retrieve the $n_{\text{sim}}$ nearest neighbors to $\ma
					q$ in a spatio-temporal volume around $\ma q$.
					\structure{The distance is computed between the basic estimates
						$\widetilde{\ma p}^{(1)}$}
					\STATE Compute $\widehat{\overline{\ma p}}\,\!^{(2)}$ and $\widehat
							 C_{r}^{(2)}$ \structure{from the basic estimates
								 $\widetilde{\ma p}_i^{(1)}$, \textbf{setting} $\ma{\sigma = 0}$.}
					\FORALL{$n_{\text{sim}}$ neighbors $\ma q_i$ of $\ma q$}
						\STATE Obtain the MAP estimate $\widetilde{\ma p}_i^{(2)}$
						\STATE Aggregate estimated patch on $\widetilde u^{(2)}$
						\STATE Remove $\ma q_i$ from $\mathcal P$
					\ENDFOR
				\ENDWHILE
	%			\FORALL{pixel $(x,t)$ in $\Omega\times {1,\dots,T}$}
	%			\STATE Obtain the basic estimate $\widetilde u^{(1)}(x,t)$ by averaging
	%			the values of all patches $\widetilde{\ma p}^{(1)}$ containing $(x,t)$.
	%			\ENDFOR
			\end{algorithmic}
			\end{spacing}
		\end{algorithm}

	\end{overprint}
	\end{column}%
%	\begin{column}{0.4\textwidth}
%	\end{column}
	\end{columns}

\end{frame}

% \begin{frame}{Note: learning in 2nd stage}
% 
% 	In the second stage the basic estimate is used as an oracle to estimate
% 	patch distances and to learn the parameters of the Gaussian linear models.
% 
% 	\bigskip
% 
% 	Since the set of patches from the basic estimate $\ma p^{(1)}_i$ are considered to
% 	be noiseless, the maximum likelihood parameters are as follows:
% 
% 	\begin{equation*}
% 		\widehat{\overline{\ma p}}\,\!^{(2)} = \frac1n\sum_{i = 1}^n \ma p^{(1)}_i
% 		\quad\quad\text{and}\quad\quad\quad
% 		\widehat{C}^{(2)} = \frac1n\sum_{i = 1}^n \ma p^{(1)\,T}_i\ma p^{(1)}_i
% 	\end{equation*}
% 	
% \end{frame}


\begin{frame}{Color videos}

	We follow \reference{Lebrun et al.'13, Dabov et al.'07}.

	\bigskip

	\setbeamertemplate{itemize items}[triangle]
	
	In the first stage: 
	\begin{itemize}\itemsep=.3cm
		\item Use YUV colorspace (opponent color transform).
		\item Patch distance computed using luminance Y only.
		\item A Gaussian model is learnt for each channel (i.e. zero covariances
			between channels)
	\end{itemize}

	\bigskip

	\bigskip

	In the second stage: 
	\begin{itemize}\itemsep=.3cm
		\item Patch distance computed using the three RGB color channels.
		\item A Gaussian model is learnt jointly for the RGB patches.
	\end{itemize}
	
\end{frame}

% \begin{frame}{Results}
% 
% 	\begin{center}
% 	Experiments on 5 sequences of the Middlebury optical flow dataset:
% 
% 	\bigskip
% 
% 	\emph{Army}, \\
% 	\emph{DogDance}, \\
% 	\emph{Evergreen}, \\
% 	\emph{Mequon}, \\
% 	\emph{Walking},
% 
% 	\bigskip
% 
% 	with white additive Gaussian noise of 
% 
% 	\bigskip
% 	
% 	$\sigma = 10$, \\ 
% 	$\sigma = 20$, \\ 
% 	$\sigma = 40$.
% 	\end{center}
% 
% \end{frame}



\begin{frame}{Comparison with V-BM4D}

	We compare with V-BM4D \reference{Maggioni et al.'12}.

	\setbeamertemplate{itemize items}[triangle]

	\begin{itemize}\itemsep=.3cm
		\item Two stages comprised of \emph{building of patch trajectories}, \emph{grouping}, \emph{collaborative
			filtering}, and \emph{aggregation}.
		\item \structure{Patch trajectories} built using block matching.
		\item \structure{Grouping:} for each spatio-temporal patch $\ma q_0$, a
			group of similar patches is build as
			\[\mathcal G(\ma q_0) = \{\ma  q \text{ spatio-temporal patch}\,:\, d(\ma q_0, \ma q) < \tau_{\text{match}}\}\]
		\item \structure{Collaborative filtering:} Each group $\mathcal G(\ma q)$ of similar 3D
			patches is treated as a 4D signal which is filtered in a transform
			domain, via thresholding (1st stage) or Wiener filtering (2nd stage). 
	\end{itemize}

\end{frame}



\begin{frame}{Parameters}
	\begin{center}

	\begin{tabular}{l | c c | c c | c c }
		& \multicolumn{2}{c|}{$\sigma = 10$} 
		& \multicolumn{2}{c|}{$\sigma = 20$} 
		& \multicolumn{2}{c}{$\sigma = 40$} \\
		                            & 1st    & 2nd   & 1st   & 2nd   & 1st   & 2nd \\\hline\hline
		Patch size (spatial)        &  7 / 5 & 7 / 5 & 7 / 5 & 7 / 5 & 7 / 5 & 7 / 5 \\
		Patch size (temporal)       &  1--4  & 1--4  & 1--4  & 1--4  & 1--4  & 1--4  \\
		Spatial search window       & 23 / 13&45 / 37&23 / 37&45 / 37&45 / 37& 45 / 37\\
		Temporal search range       & 5    & 5    & 5    & 5    & 5    & 5   \\
		Number of patches           & 225 / 375  & 160  & 225 / 375 & 160  & 450 / 375 & 160 \\
		Rank                        & full & 16   & full & 16   & full & 16  \\
%		Distance threshold $\tau$   & n/a  & 432  & n/a  & 432  & n/a  & 432 \\
%		Beta                        & 1    & 1.2  & 1    & 1.2  & 1    & 1.2 \\\hline
	\end{tabular}

	\bigskip

	We write \emph{grayscale-parameters / color-parameter} in case\\ a parameter
	differs between grascale and color videos.

	\pause
	\bigskip

	\begin{tabular}{l | c c}
		                              & 1st & 2nd  \\\hline\hline
		Patch size (spatial)          &  8  &   8  \\
		Patch size (temporal)         & 15  &  15  \\
		Spatial search window         & 15  &  15  \\
		Temporal search range         &  1  &   1  \\
		Number of patches             & 32  &  32  \\
		Spatial search window (M.E.)  & 11  &  11  \\
	\end{tabular}

	\vspace{.1cm}
	Parameters of V-BM4D (M.E.: motion estimation)

	\end{center}
\end{frame}

% \multipleframe
% \begin{frame}{PSNRs of basic estimates}
% 
% 
% 	\begin{center}
% 	{\tiny
% 	\renewcommand{\tabcolsep}{2mm}
% 	\renewcommand{\arraystretch}{1.0}
% 	\begin{tabular}{ c | l |c c c c c}
% 		% NOTE: These results were obtained with the ACIVS paramter set. These
% 		% parameters were tuned to the DERF database, but perform a little worse in
% 		% the Middleburry sequences.
% 		\hline
% 		\rule{0pt}{6pt}$\sigma$ & Method           & Army & DogDance & Evergreen & Mequon & Walking  \\\hline
% 		\multirow{5}{*}{$10$} & V-BM4D-np        & \best{37.48} & \best{35.60} & \best{35.29} & \best{37.89} & \best{38.67} \\%\cline{2-7}
% %		                      & VNLB2D $n_t = 0$ &       34.98  &       34.03  &       33.60  &       35.93  &       36.19  \\%\cline{2-7}
% %		                      & VNLB2D $n_t = 1$ &       36.49  &       35.02  &       34.56  &       37.65  &       38.06  \\%\cline{2-7}
% 		                      & VNLB2D $n_t = 5$ &       36.66  &       35.10  &       34.64  &       37.74  &       38.30  \\
% 		                      & VNLB   $s_t = 1$ &       todo   &       todo   &       todo   &       todo   &       todo   \\
% 		                      & VNLB   $s_t = 2$ &       37.68  &       35.57  &       35.34  &       37.72  &       38.54  \\
% 		                      & VNLB   $s_t = 3$ &       37.60  &       35.58  &       35.33  &       37.37  &       38.18  \\
% 		                      & VNLB   $s_t = 4$ &       37.20  &       35.38  &       35.16  &       36.99  &       37.67  \\\hline
% %
% 		\multirow{5}{*}{$20$} & V-BM4D-np        & \best{32.24} & \best{31.07} & \best{30.48} &       32.89  & \best{33.54} \\%\cline{2-7}
% %		                      & VNLB2D $n_t = 0$ &       31.45  &       30.75  &       30.28  &       32.29  &       32.43  \\%\cline{2-7}
% %		                      & VNLB2D $n_t = 1$ &       33.13  &       31.95  &       31.42  &       34.15  &       34.35  \\%\cline{2-7}
% 		                      & VNLB2D $n_t = 5$ &       33.38  &       32.09  &       31.56  &       34.39  &       34.64  \\
% 		                      & VNLB   $s_t = 1$ &       todo   &       todo   &       todo   &       todo   &       todo   \\
% 		                      & VNLB   $s_t = 2$ &       34.43  &       32.70  &       32.29  &       34.73  &       35.30  \\
% 		                      & VNLB   $s_t = 3$ &       34.42  &       32.71  &       32.31  &       34.50  &       35.14  \\
% 		                      & VNLB   $s_t = 4$ &       34.09  &       32.53  &       32.15  &       34.16  &       34.79  \\\hline
% %
% 		\multirow{5}{*}{$40$} & V-BM4D-np        & \best{29.53} & \best{28.59} &       27.93  &       29.94  & \best{30.55} \\%\cline{2-7}
% %		                      & VNLB2D $n_t = 0$ &       26.91  &       26.53  &       26.19  &       27.45  &       27.53  \\%\cline{2-7}
% %		                      & VNLB2D $n_t = 1$ &       29.06  &       28.25  &       27.84  &       29.72  &       29.84  \\%\cline{2-7}
% 		                      & VNLB2D $n_t = 5$ &       29.44  &       28.52  &       28.10  &       30.09  &       30.24  \\
% 		                      & VNLB   $s_t = 1$ &       todo   &       todo   &       todo   &       todo   &       todo   \\
% 		                      & VNLB   $s_t = 2$ &       30.49  &       29.26  &       28.88  &       30.66  &       31.07  \\
% 		                      & VNLB   $s_t = 3$ &       30.44  &       29.24  &       28.86  &       30.46  &       30.97  \\
% 		                      & VNLB   $s_t = 4$ &       30.06  &       29.00  &       28.61  &       30.09  &       30.65  \\\hline
% 		\end{tabular}}
% 
% 	\bigskip
% 
% 	PSNRs obtained for the five color sequences from the Middlebury dataset.
% 
% 	\end{center}
% 
% %	\begin{center}
% %	\begin{tabular}{| c | c |c c c c c|}
% %		\hline \hline
% %		$\sigma$  & Method \textbackslash Video & Army & DogDance & Evergreen & Mequon & Walking \\\hline\hline
% %		\multirow{5}{*}{$10$} & BM4D           & \best{37.48} & \best{35.60} & \best{35.29} & \best{37.89} & \best{38.67} \\%\cline{2-7}
% %		                      & VNLB $W_t = 0$ &       14.30  &       18.98  &       15.87  &       16.83  &       16.28  \\%\cline{2-7}
% %		                      & VNLB $W_t = 1$ &       36.21  &       34.86  &       34.50  &       37.16  &       37.54  \\%\cline{2-7}
% %		                      & VNLB $W_t = 2$ &       36.77  &       35.22  &       34.82  &       37.69  &       38.20  \\\hline
% %%
% %		\multirow{5}{*}{$25$} & BM4D           & \best{32.24} & \best{31.07} & \best{30.48} &       32.89  & \best{33.54} \\%\cline{2-7}
% %		                      & VNLB $W_t = 0$ &       30.13  &       29.51  &       29.07  &       30.91  &       30.00  \\%\cline{2-7}
% %		                      & VNLB $W_t = 1$ &       31.92  &       30.85  &       30.33  &       32.86  &       33.00  \\%\cline{2-7}
% %		                      & VNLB $W_t = 2$ & \best{32.21} & \best{31.02} & \best{30.50} & \best{33.14} &       33.31  \\\hline
% %%
% %		\multirow{5}{*}{$40$} & BM4D           & \best{29.53} & \best{28.59} &       27.93  &       29.94  & \best{30.55} \\%\cline{2-7}
% %		                      & VNLB $W_t = 0$ &       27.03  &       26.62  &       26.27  &       27.59  &       27.65  \\%\cline{2-7}
% %		                      & VNLB $W_t = 1$ &       29.13  &       28.30  &       27.88  &       29.79  &       29.91  \\%\cline{2-7}
% %		                      & VNLB $W_t = 2$ & \best{29.50} & \best{28.55} & \best{28.13} & \best{30.15} &       30.29  \\\hline\hline
% %	\end{tabular}
% %
% %	\bigskip
% %	Results of VNLB corresponding to patch sizes of $5$ in both steps.\\The rest of parameters are the default ones.
% %	\end{center}
% 
% \end{frame}
% 
% \begin{frame}{PSNRs of final estimates}
% 
% 
% 	\begin{center}
% 	{\tiny
% 	\renewcommand{\tabcolsep}{2mm}
% 	\renewcommand{\arraystretch}{1.0}
% 	\begin{tabular}{ c | l |c c c c c}
% 		% NOTE: These results were obtained with the ACIVS paramter set. These
% 		% parameters were tuned to the DERF database, but perform a little worse in
% 		% the Middleburry sequences.
% 		\hline
% 		\rule{0pt}{6pt}$\sigma$ & Method        & Army & DogDance & Evergreen & Mequon & Walking  \\\hline
% 		\multirow{5}{*}{$10$} & V-BM4D-np        &       37.77  &       35.70  &       35.40  &       38.09  &       38.85  \\
% %		                      & VNLB2D $n_t = 0$ &       36.90  &       35.42  &       34.84  &       38.43  &       38.86  \\
% %		                      & VNLB2D $n_t = 3$ &       37.65  &       35.87  &       35.38  &       39.09  &       39.69  \\
% 		                      & VNLB2D $n_t = 5$ & \best{37.88} & \best{36.01} & \best{35.53} & \Best{39.20} & \best{39.74} \\
% 		                      & VNLB   $s_t = 1$ &       todo   &       todo   &       todo   &       todo   &       todo   \\
% 		                      & VNLB   $s_t = 2$ &       38.69  &       36.26  &       35.97  & \best{39.02} & \Best{39.95} \\
% 		                      & VNLB   $s_t = 3$ &       39.06  & \Best{36.48} & \Best{36.20} & \best{38.96} & \Best{39.97} \\
% 		                      & VNLB   $s_t = 4$ & \Best{39.17} & \Best{36.55} & \Best{36.29} &       38.86  & \Best{39.90} \\\hline
% %
% 		\multirow{5}{*}{$20$} & V-BM4D-np        &       32.91  &       31.50  &       30.94  &       33.60  &       34.27  \\
% %		                      & VNLB2D $n_t = 0$ &       33.77  &       32.44  &       31.68  &       35.61  &       35.60  \\
% %		                      & VNLB2D $n_t = 3$ &       34.45  &       32.90  &       32.15  &       35.89  &       36.12  \\
% 		                      & VNLB2D $n_t = 5$ & \best{34.59} & \best{33.02} & \best{32.27} & \best{35.90} & \best{36.12} \\
% 		                      & VNLB   $s_t = 1$ &       todo   &       todo   &       todo   &       todo   &       todo   \\
% 		                      & VNLB   $s_t = 2$ &       35.55  &       33.37  &       32.78  & \Best{36.19} &       36.95  \\
% 		                      & VNLB   $s_t = 3$ &       35.94  & \Best{33.58} &       33.06  & \Best{36.13} & \Best{37.07} \\
% 		                      & VNLB   $s_t = 4$ & \Best{36.08} & \Best{33.65} & \Best{33.20} &       36.04  & \Best{37.05} \\\hline
% %
% 		\multirow{5}{*}{$40$} & V-BM4D-np        &       30.42  &       29.29  &       28.65  &       31.02  &       31.64  \\
% %		                      & VNLB2D $n_t = 0$ &       30.55  &       29.41  &       28.66  &       31.78  &       31.86  \\
% %		                      & VNLB2D $n_t = 3$ & \best{31.02} & \best{29.73} & \best{29.03} & \best{31.88} & \best{31.92} \\
% 		                      & VNLB2D $n_t = 5$ & \best{31.06} & \best{29.74} & \best{29.08} & \best{31.70} & \best{31.74} \\
% 		                      & VNLB   $s_t = 1$ &       todo   &       todo   &       todo   &       todo   &       todo   \\
% 		                      & VNLB   $s_t = 2$ &       32.60  &       30.63  &       29.91  & \Best{33.00} &       33.63  \\
% 		                      & VNLB   $s_t = 3$ &       33.03  & \Best{30.87} &       30.20  & \Best{33.05} & \Best{33.89} \\
% 		                      & VNLB   $s_t = 4$ & \Best{33.17} & \Best{30.95} & \Best{30.34} & \Best{33.00} & \Best{33.95} \\\hline
% %
% 		\end{tabular}}
% 
% 	\bigskip
% 	
% 	PSNRs obtained for the five color sequences from the Middlebury dataset.
% 
% 	\end{center}
% 
% 
% 
% 
% 
% %	\begin{center}
% %	\begin{tabular}{| c | c |c c c c c|}
% %		% NOTE: These results were obtained with a set of experiments run in
% %		order to tune the parametrs.
% %		\hline \hline
% %		$\sigma$  & Method \textbackslash Video & Army & DogDance & Evergreen & Mequon & Walking \\\hline\hline
% %		\multirow{5}{*}{$10$} & BM4D           & \best{37.77} &       35.70  & \best{35.40} &       38.09  &       38.85  \\%\cline{2-7}
% %		                      & VNLB $W_t = 0$ &       34.46  &       33.61  &       33.05  &       35.65  &       35.90  \\%\cline{2-7}
% %		                      & VNLB $W_t = 1$ &       37.65  & \best{35.87} & \best{35.42} & \best{39.10} & \best{39.70} \\%\cline{2-7}
% %		                      & VNLB $W_t = 2$ & \best{37.75} & \best{35.92} & \best{35.47} & \best{39.06} & \best{39.79} \\\hline
% %%
% %		\multirow{5}{*}{$25$} & BM4D           &       32.91  &       31.50  &       30.94  &       33.60  &       34.27  \\%\cline{2-7}
% %		                      & VNLB $W_t = 0$ &       32.33  &       31.17  &       30.39  &       33.69  &       33.74  \\%\cline{2-7}
% %		                      & VNLB $W_t = 1$ &       33.48  & \best{31.98} & \best{31.22} & \best{35.10} &       35.25  \\%\cline{2-7}
% %		                      & VNLB $W_t = 2$ & \best{33.58} & \best{31.98} & \best{31.22} & \best{35.08} & \best{35.38} \\\hline
% %%
% %		\multirow{5}{*}{$40$} & BM4D           &       30.42  &       29.29  &       28.65  &       31.02  &       31.64  \\%\cline{2-7}
% %		                      & VNLB $W_t = 0$ &       29.94  &       28.95  &       28.24  &       30.98  &       30.95  \\%\cline{2-7}
% %		                      & VNLB $W_t = 1$ &       31.36  &       29.96  & \best{29.19} & \best{32.70} &       32.71  \\%\cline{2-7}
% %		                      & VNLB $W_t = 2$ & \best{31.55} & \best{30.01} & \best{29.24} & \best{32.75} & \best{32.94} \\\hline\hline
% %	\end{tabular}
% %
% %	\bigskip
% %	Results of VNLB corresponding to patch sizes of $5$ in both steps.\\The rest of parameters are the default ones.
% %	\end{center}
% 
% \end{frame}
% \restoreframe


% % results overview on Army
% \multipleframe
% \begin{frame}{Overview of results}
% 	\begin{center}
%  			\animategraphics[palindrome, autoplay, width=0.9\textwidth]{10}{figures/combined_videos/Army_s40/orig-nisy_}{000}{007}
% 	\end{center}
% 
% 	\begin{center}
% 		\emph{Army}. Original and noisy ($\sigma = 40$), on a checkerboard pattern.
% 	\end{center}
% \end{frame}
% \begin{frame}{Overview of results}
% 	\begin{center}
%  			\animategraphics[palindrome, autoplay, width=0.9\textwidth]{10}{figures/combined_videos/Army_s40/vnlb_pt1-vnlb_pt1_}{000}{007}
% 	\end{center}
% 
% 	\begin{center}
% 		\emph{Army}. Result using VNLB with $s_t = 1$.
% 	\end{center}
% \end{frame}
% \begin{frame}{Overview of results}
% 	\begin{center}
%  			\animategraphics[palindrome, autoplay, width=0.9\textwidth]{10}{figures/combined_videos/Army_s40/vnlb_pt2-vnlb_pt1_}{000}{007}
% 	\end{center}
% 
% 	\begin{center}
% 		\emph{Army}. Results using VNLB with $s_t = 2$ and $s_t = 1$, on a checkerboard pattern.
% 	\end{center}
% \end{frame}
% \begin{frame}{Overview of results}
% 	\begin{center}
%  			\animategraphics[palindrome, autoplay, width=0.9\textwidth]{10}{figures/combined_videos/Army_s40/vnlb_pt3-vnlb_pt1_}{000}{007}
% 	\end{center}
% 
% 	\begin{center}
% 		\emph{Army}. Results using VNLB with $s_t = 3$ and $s_t = 1$, on a checkerboard pattern.
% 	\end{center}
% \end{frame}
% \begin{frame}{Overview of results}
% 	\begin{center}
%  			\animategraphics[palindrome, autoplay, width=0.9\textwidth]{10}{figures/combined_videos/Army_s40/vnlb_pt4-vnlb_pt1_}{000}{007}
% 	\end{center}
% 
% 	\begin{center}
% 		\emph{Army}. Results using VNLB with $s_t = 4$ and $s_t = 1$, on a checkerboard pattern.
% 	\end{center}
% \end{frame}
% \begin{frame}{Overview of results}
% 	\begin{center}
%  			\animategraphics[palindrome, autoplay, width=0.9\textwidth]{10}{figures/combined_videos/Army_s40/vnlb_pt4-bm4d_}{000}{007}
% 	\end{center}
% 
% 	\begin{center}
% 		\emph{Army}. Results using VNLB with $s_t = 4$ and BM4D, on a checkerboard pattern.
% 	\end{center}
% \end{frame}
% \begin{frame}{Overview of results}
% 	\begin{center}
%  			\animategraphics[palindrome, autoplay, width=0.9\textwidth]{10}{figures/combined_videos/Army_s40/vnlb_pt4-orig_}{000}{007}
% 	\end{center}
% 
% 	\begin{center}
% 		\emph{Army}. Results using VNLB with $s_t = 4$ and original, on a checkerboard pattern.
% 	\end{center}
% \end{frame}
% \restoreframe

\multipleframe
% \begin{frame}{Results on grayscale sequences}
% 	\begin{center}
%  			\animategraphics[height=5cm]{30}{derf/mobile_mono/}{001}{300}\\
% 			Mobile: original sequence
% 	\end{center}
% \end{frame}
% 
% \begin{frame}{Results on grayscale sequences}
% 	\begin{center}
%  			\animategraphics[height=5cm]{30}{vnlb3d_mono/mobile_mono_s20_pt4/nisy_}{001}{300}\\
% 			Mobile: AWGN with $\sigma = 20$ added
% 	\end{center}
% \end{frame}
% 
% \begin{frame}{Results on grayscale sequences}
% 	\begin{center}
%  			\animategraphics[height=5cm]{30}{VBM4D/mobile_mono_mp_s20/deno_}{001}{300}\\
% 			Mobile: result of V-BM4D-mp
% 	\end{center}
% \end{frame}
% 
% \begin{frame}{Results on grayscale sequences}
% 	\begin{center}
%  			\animategraphics[height=5cm]{30}{vnlb3d_mono/mobile_mono_s20_pt4/deno_}{001}{300}\\
% 			Mobile: result of VNLB  
% 	\end{center}
% \end{frame}

% mobile detail1
\begin{frame}{Results on grayscale sequences}
	\begin{center}
% 			\animategraphics[palindrome, controls, autopause, height=4cm]{10}{VBM4D/mobile_mono_mp_s10/tile060-060-080_}{001}{010}
% 			\animategraphics[palindrome, controls, autopause, height=4cm]{10}{vnlb3d_mono/mobile_mono_s10_pt4/tile060-060-080_}{001}{010}
%			\begin{animateinline}[palindrome, controls, autopause]{10}
			\begin{animateinline}[palindrome, autoplay]{10}
				\multiframe{30}{i=1+1}{%
					\begin{tabular}{c}
					\includegraphics[height=3.5cm]{data/derf/mobile_mono/tile060-060-080_\zeropad{123}{\i}}$\,$ 
					\includegraphics[height=3.5cm]{vnlb3d_mono/mobile_mono_s10_pt4/nisy_tile060-060-080_\zeropad{123}{\i}}\\
					\includegraphics[height=3.5cm]{VBM4D/mobile_mono_mp_s10/tile060-060-080_\zeropad{123}{\i}}$\,$ 
					\includegraphics[height=3.5cm]{vnlb3d_mono/mobile_mono_s10_pt4/tile060-060-080_\zeropad{123}{\i}}
				\end{tabular}
				}
			\end{animateinline}
	\end{center}

	\begin{center}
		\emph{Mobile} (detail). Top: original and noisy ($\sigma = 10$), bottom: V-BM4D-mp and VNLB.
	\end{center}
\end{frame}
\begin{frame}{Results on grayscale sequences}
	\begin{center}
% 			\animategraphics[palindrome, controls, autopause, height=4cm]{10}{VBM4D/mobile_mono_mp_s20/tile060-060-080_}{001}{010}
% 			\animategraphics[palindrome, controls, autopause, height=4cm]{10}{vnlb3d_mono/mobile_mono_s20_pt4/tile060-060-080_}{001}{010}
%			\begin{animateinline}[palindrome, controls, autopause]{10}
			\begin{animateinline}[palindrome, autoplay]{10}
				\multiframe{30}{i=1+1}{%
					\begin{tabular}{c}
					\includegraphics[height=3.5cm]{data/derf/mobile_mono/tile060-060-080_\zeropad{123}{\i}}$\,$ 
					\includegraphics[height=3.5cm]{vnlb3d_mono/mobile_mono_s20_pt4/nisy_tile060-060-080_\zeropad{123}{\i}}\\
					\includegraphics[height=3.5cm]{VBM4D/mobile_mono_mp_s20/tile060-060-080_\zeropad{123}{\i}}$\,$ 
					\includegraphics[height=3.5cm]{vnlb3d_mono/mobile_mono_s20_pt4/tile060-060-080_\zeropad{123}{\i}}
				\end{tabular}
				}
			\end{animateinline}
	\end{center}

	\begin{center}
		\emph{Mobile} (detail). Top: original and noisy ($\sigma = 20$), bottom: V-BM4D-mp and VNLB.
	\end{center}
\end{frame}
\begin{frame}{Results on grayscale sequences}
	\begin{center}
% 			\animategraphics[palindrome, controls, autopause, height=4cm]{10}{VBM4D/mobile_mono_mp_s40/tile060-060-080_}{001}{010}
% 			\animategraphics[palindrome, controls, autopause, height=4cm]{10}{vnlb3d_mono/mobile_mono_s40_pt4/tile060-060-080_}{001}{010}
%			\begin{animateinline}[palindrome, controls, autopause]{10}
			\begin{animateinline}[palindrome, autoplay]{10}
				\multiframe{30}{i=1+1}{%
				\begin{tabular}{c}
					\includegraphics[height=3.5cm]{data/derf/mobile_mono/tile060-060-080_\zeropad{123}{\i}}$\,$ 
					\includegraphics[height=3.5cm]{vnlb3d_mono/mobile_mono_s40_pt4/nisy_tile060-060-080_\zeropad{123}{\i}}\\
					\includegraphics[height=3.5cm]{VBM4D/mobile_mono_mp_s40/tile060-060-080_\zeropad{123}{\i}}$\,$ 
					\includegraphics[height=3.5cm]{vnlb3d_mono/mobile_mono_s40_pt4/tile060-060-080_\zeropad{123}{\i}}
				\end{tabular}
				}
			\end{animateinline}
	\end{center}

	\begin{center}
		\emph{Mobile} (detail). Top: original and noisy ($\sigma = 40$), bottom: V-BM4D-mp and VNLB.
	\end{center}
\end{frame}

% mobile detail2
\begin{frame}{Results on grayscale sequences}
	\begin{center}
% 			\animategraphics[palindrome, controls, autopause, height=4cm]{10}{VBM4D/mobile_mono_mp_s10/tile188-188-065_}{001}{010}
% 			\animategraphics[palindrome, controls, autopause, height=4cm]{10}{vnlb3d_mono/mobile_mono_s10_pt4/tile188-188-065_}{001}{010}
%			\begin{animateinline}[palindrome, controls, autopause]{10}
			\begin{animateinline}[palindrome, autoplay]{10}
				\multiframe{30}{i=1+1}{%
				\begin{tabular}{c}
					\includegraphics[height=3.5cm]{data/derf/mobile_mono/tile188-188-065_\zeropad{123}{\i}}$\,$ 
					\includegraphics[height=3.5cm]{vnlb3d_mono/mobile_mono_s10_pt4/nisy_tile188-188-065_\zeropad{123}{\i}}\\
					\includegraphics[height=3.5cm]{VBM4D/mobile_mono_mp_s10/tile188-188-065_\zeropad{123}{\i}}$\,$ 
					\includegraphics[height=3.5cm]{vnlb3d_mono/mobile_mono_s10_pt4/tile188-188-065_\zeropad{123}{\i}}
				\end{tabular}
				}
			\end{animateinline}
	\end{center}

	\begin{center}
		\emph{Mobile} (detail). Top: original and noisy ($\sigma = 10$), bottom: V-BM4D-mp and VNLB.
	\end{center}
\end{frame}
\begin{frame}{Results on grayscale sequences}
	\begin{center}
% 			\animategraphics[palindrome, controls, autopause, height=4cm]{10}{VBM4D/mobile_mono_mp_s20/tile188-188-065_}{001}{010}
% 			\animategraphics[palindrome, controls, autopause, height=4cm]{10}{vnlb3d_mono/mobile_mono_s20_pt4/tile188-188-065_}{001}{010}
%			\begin{animateinline}[palindrome, controls, autopause]{10}
			\begin{animateinline}[palindrome, autoplay]{10}
				\multiframe{30}{i=1+1}{%
				\begin{tabular}{c}
					\includegraphics[height=3.5cm]{data/derf/mobile_mono/tile188-188-065_\zeropad{123}{\i}}$\,$ 
					\includegraphics[height=3.5cm]{vnlb3d_mono/mobile_mono_s20_pt4/nisy_tile188-188-065_\zeropad{123}{\i}}\\
					\includegraphics[height=3.5cm]{VBM4D/mobile_mono_mp_s20/tile188-188-065_\zeropad{123}{\i}}$\,$ 
					\includegraphics[height=3.5cm]{vnlb3d_mono/mobile_mono_s20_pt4/tile188-188-065_\zeropad{123}{\i}}
				\end{tabular}
				}
			\end{animateinline}
	\end{center}

	\begin{center}
		\emph{Mobile} (detail). Top: original and noisy ($\sigma = 20$), bottom: V-BM4D-mp and VNLB.
	\end{center}
\end{frame}
\begin{frame}{Results on grayscale sequences}
	\begin{center}
% 			\animategraphics[palindrome, controls, autopause, height=4cm]{10}{VBM4D/mobile_mono_mp_s40/tile188-188-065_}{001}{010}
% 			\animategraphics[palindrome, controls, autopause, height=4cm]{10}{vnlb3d_mono/mobile_mono_s40_pt4/tile188-188-065_}{001}{010}
%			\begin{animateinline}[palindrome, controls, autopause]{10}
			\begin{animateinline}[palindrome, autoplay]{10}
				\multiframe{30}{i=1+1}{%
				\begin{tabular}{c}
					\includegraphics[height=3.5cm]{data/derf/mobile_mono/tile188-188-065_\zeropad{123}{\i}}$\,$ 
					\includegraphics[height=3.5cm]{vnlb3d_mono/mobile_mono_s40_pt4/nisy_tile188-188-065_\zeropad{123}{\i}}\\
					\includegraphics[height=3.5cm]{VBM4D/mobile_mono_mp_s40/tile188-188-065_\zeropad{123}{\i}}$\,$ 
					\includegraphics[height=3.5cm]{vnlb3d_mono/mobile_mono_s40_pt4/tile188-188-065_\zeropad{123}{\i}}
				\end{tabular}
				}
			\end{animateinline}
	\end{center}

	\begin{center}
		\emph{Mobile} (detail). Top: original and noisy ($\sigma = 40$), bottom: V-BM4D-mp and VNLB.
	\end{center}
\end{frame}
\restoreframe


\begin{frame}{PSNRs for grayscale test sequences}
	\begin{center}
		{\small
		\renewcommand{\tabcolsep}{2mm}
		\renewcommand{\arraystretch}{1.0}
		\begin{tabular}{ c | l |c c | c c | c c | c c}
			\hline
			\rule{0pt}{6pt}$\sigma$ & Method             & \multicolumn{2}{c}{Tennis}  & \multicolumn{2}{c}{Mobile}  &\multicolumn{2}{c}{Stefan}   & \multicolumn{2}{c}{Football} \\\hline
			\multirow{5}{*}{$10$} & V-BM4D-mp            & \bsic{34.32} &       34.95  & \bsic{33.99} &       34.11  & \bsic{33.47} &       33.68  & \bsic{34.22} &       34.95  \\
%			                      & VNLB   $s_t = 1$     & \bsic{todo } &       todo   & \bsic{todo } &       todo   & \bsic{todo } &       todo   & \bsic{todo } &       todo   \\
			                      & VNLB   $s_t = 2$     & \bsic{34.33} &       34.62  & \bsic{34.44} &       34.67  & \Bsic{33.82} & \Best{34.23} & \Bsic{34.91} &       35.61  \\
										 & VNLB   $s_t = 3$     & \Bsic{34.53} &       35.15  & \Bsic{34.74} &       35.31  & \Bsic{33.81} & \Best{34.29} & \bsic{34.70} & \Best{35.67} \\
			                      & VNLB   $s_t = 4$     & \bsic{34.18} & \Best{35.32} & \bsic{34.41} & \Best{35.54} & \bsic{33.47} &       34.13  & \bsic{34.18} &       35.61  \\\hline
%
			\multirow{5}{*}{$20$} & V-BM4D-mp            & \bsic{30.54} &       31.08  & \bsic{29.67} &       30.49  & \bsic{29.04} &       29.69  & \bsic{30.16} &       31.06  \\
%			                      & VNLB   $s_t = 1$     & \bsic{todo } &       todo   & \bsic{todo } &       todo   & \bsic{todo } &       todo   & \bsic{todo } &       todo   \\
			                      & VNLB   $s_t = 2$     & \Bsic{30.73} &       31.24  & \bsic{29.83} &       30.34  & \Bsic{29.38} &       29.90  & \Bsic{30.96} & \Best{31.82} \\
										 & VNLB   $s_t = 3$     & \Bsic{30.82} &       31.66  & \Bsic{30.15} &       31.09  & \Bsic{29.33} & \Best{30.02} & \bsic{30.67} & \Best{31.83} \\
			                      & VNLB   $s_t = 4$     & \bsic{30.25} & \Best{31.69} & \bsic{29.74} & \Best{31.35} & \bsic{28.89} & \Best{29.97} & \bsic{29.96} &       31.66  \\\hline
%
			\multirow{5}{*}{$40$} & V-BM4D-mp            & \bsic{27.67} &       28.38  & \bsic{24.64} &       26.02  & \bsic{24.47} &       25.64  & \bsic{26.67} &       27.62  \\
%			                      & VNLB   $s_t = 1$     & \bsic{todo } &       todo   & \bsic{todo } &       todo   & \bsic{todo } &       todo   & \bsic{todo } &       todo   \\
			                      & VNLB   $s_t = 2$     & \bsic{27.14} &       28.15  & \bsic{25.43} &       26.06  & \bsic{25.18} &       25.81  & \bsic{27.23} &       28.24  \\
										 & VNLB   $s_t = 3$     & \bsic{27.63} &       28.62  & \bsic{26.15} &       26.94  & \Bsic{25.44} & \Best{26.12} & \Bsic{27.37} & \Best{28.42} \\
			                      & VNLB   $s_t = 4$     & \Bsic{27.81} & \Best{28.86} & \Bsic{26.45} & \Best{27.46} & \Bsic{25.50} & \Best{26.24} & \Bsic{27.31} & \Best{28.45} \\\hline
		\end{tabular}}
% Command to print rounded psnrs
% for i in $(cat bus_s40_pt*/measures | grep PSNR_final | sed "s/^-PSNR_final\ =\ "//); do  echo "scale=2;(((10^2)*$i)+0.5)/(10^2)" | bc; done

		\bigskip

		PSNRs obtained for the four classic color test sequences.
	\end{center}
\end{frame}


\multipleframe
% tennis detail1 
\begin{frame}{Results on color sequences}
	\begin{center}
% 			\animategraphics[palindrome, controls, autopause, height=4cm]{10}{VBM4D/tennis_mp_s10/tile050-090-030_}{001}{030}
% 			\animategraphics[palindrome, controls, autopause, height=4cm]{10}{vnlb3d_color/tennis_s10_pt4/tile050-090-030_}{001}{030}
			\begin{animateinline}[palindrome, autoplay]{10}
				\multiframe{30}{i=1+1}{%
				\begin{tabular}{c}
					\includegraphics[height=3.5cm]{data/derf/tennis/tile050-090-030_\zeropad{123}{\i}}$\,$ 
					\includegraphics[height=3.5cm]{vnlb3d_color/tennis_s10_pt4/nisy_tile050-090-030_\zeropad{123}{\i}}\\
					\includegraphics[height=3.5cm]{VBM4D/tennis_mp_s10/tile050-090-030_\zeropad{123}{\i}}$\,$ 
					\includegraphics[height=3.5cm]{vnlb3d_color/tennis_s10_pt4/tile050-090-030_\zeropad{123}{\i}}
				\end{tabular}
				}
			\end{animateinline}
	\end{center}

	\begin{center}
		\emph{Tennis} (detail). Top: original and noisy ($\sigma = 10$), bottom: V-BM4D-mp and VNLB.
	\end{center}
\end{frame}
\begin{frame}{Results on color sequences}
	\begin{center}
% 			\animategraphics[palindrome, controls, autopause, height=4cm]{10}{VBM4D/tennis_mp_s20/tile050-090-030_}{001}{030}
% 			\animategraphics[palindrome, controls, autopause, height=4cm]{10}{vnlb3d_color/tennis_s20_pt4/tile050-090-030_}{001}{030}
			\begin{animateinline}[palindrome, autoplay]{10}
				\multiframe{30}{i=1+1}{%
				\begin{tabular}{c}
					\includegraphics[height=3.5cm]{data/derf/tennis/tile050-090-030_\zeropad{123}{\i}}$\,$ 
					\includegraphics[height=3.5cm]{vnlb3d_color/tennis_s20_pt4/nisy_tile050-090-030_\zeropad{123}{\i}}\\
					\includegraphics[height=3.5cm]{VBM4D/tennis_mp_s20/tile050-090-030_\zeropad{123}{\i}}$\,$ 
					\includegraphics[height=3.5cm]{vnlb3d_color/tennis_s20_pt4/tile050-090-030_\zeropad{123}{\i}}
				\end{tabular}
				}
			\end{animateinline}
	\end{center}

	\begin{center}
		\emph{Tennis} (detail). Top: original and noisy ($\sigma = 20$), bottom: V-BM4D-mp and VNLB.
	\end{center}
\end{frame}
\begin{frame}{Results on color sequences}
	\begin{center}
% 			\animategraphics[palindrome, controls, autopause, height=4cm]{10}{VBM4D/tennis_mp_s40/tile050-090-030_}{001}{030}
% 			\animategraphics[palindrome, controls, autopause, height=4cm]{10}{vnlb3d_color/tennis_s40_pt4/tile050-090-030_}{001}{030}
			\begin{animateinline}[palindrome, autoplay]{10}
				\multiframe{30}{i=1+1}{%
				\begin{tabular}{c}
					\includegraphics[height=3.5cm]{data/derf/tennis/tile050-090-030_\zeropad{123}{\i}}$\,$ 
					\includegraphics[height=3.5cm]{vnlb3d_color/tennis_s40_pt4/nisy_tile050-090-030_\zeropad{123}{\i}}\\
					\includegraphics[height=3.5cm]{VBM4D/tennis_mp_s40/tile050-090-030_\zeropad{123}{\i}}$\,$ 
					\includegraphics[height=3.5cm]{vnlb3d_color/tennis_s40_pt4/tile050-090-030_\zeropad{123}{\i}}
				\end{tabular}
				}
			\end{animateinline}
	\end{center}

	\begin{center}
		\emph{Tennis} (detail). Top: original and noisy ($\sigma = 40$), bottom: V-BM4D-mp and VNLB.
	\end{center}
\end{frame}

% tennis detail2 
\begin{frame}{Results on color sequences}
	\begin{center}
% 			\animategraphics[palindrome, controls, autopause, height=4cm]{10}{VBM4D/tennis_mp_s10/tile040-150-070_}{001}{030}
% 			\animategraphics[palindrome, controls, autopause, height=4cm]{10}{vnlb3d_color/tennis_s10_pt4/tile040-150-070_}{001}{030}
			\begin{animateinline}[palindrome, autoplay]{10}
				\multiframe{30}{i=1+1}{%
				\begin{tabular}{c}
					\includegraphics[height=3.5cm]{data/derf/tennis/tile040-150-070_\zeropad{123}{\i}}$\,$ 
					\includegraphics[height=3.5cm]{vnlb3d_color/tennis_s10_pt4/nisy_tile040-150-070_\zeropad{123}{\i}}\\
					\includegraphics[height=3.5cm]{VBM4D/tennis_mp_s10/tile040-150-070_\zeropad{123}{\i}}$\,$ 
					\includegraphics[height=3.5cm]{vnlb3d_color/tennis_s10_pt4/tile040-150-070_\zeropad{123}{\i}}
				\end{tabular}
				}
			\end{animateinline}
	\end{center}

	\begin{center}
		\emph{Tennis} (detail). Top: original and noisy ($\sigma = 10$), bottom: V-BM4D-mp and VNLB.
	\end{center}
\end{frame}
\begin{frame}{Results on color sequences}
	\begin{center}
% 			\animategraphics[palindrome, controls, autopause, height=4cm]{10}{VBM4D/tennis_mp_s20/tile040-150-070_}{001}{030}
% 			\animategraphics[palindrome, controls, autopause, height=4cm]{10}{vnlb3d_color/tennis_s20_pt4/tile040-150-070_}{001}{030}
			\begin{animateinline}[palindrome, autoplay]{10}
				\multiframe{30}{i=1+1}{%
				\begin{tabular}{c}
					\includegraphics[height=3.5cm]{data/derf/tennis/tile040-150-070_\zeropad{123}{\i}}$\,$ 
					\includegraphics[height=3.5cm]{vnlb3d_color/tennis_s20_pt4/nisy_tile040-150-070_\zeropad{123}{\i}}\\
					\includegraphics[height=3.5cm]{VBM4D/tennis_mp_s20/tile040-150-070_\zeropad{123}{\i}}$\,$ 
					\includegraphics[height=3.5cm]{vnlb3d_color/tennis_s20_pt4/tile040-150-070_\zeropad{123}{\i}}
				\end{tabular}
				}
			\end{animateinline}
	\end{center}

	\begin{center}
		\emph{Tennis} (detail). Top: original and noisy ($\sigma = 20$), bottom: V-BM4D-mp and VNLB.
	\end{center}
\end{frame}
\begin{frame}{Results on color sequences}
	\begin{center}
% 			\animategraphics[palindrome, controls, autopause, height=4cm]{10}{VBM4D/tennis_mp_s40/tile040-150-070_}{001}{030}
% 			\animategraphics[palindrome, controls, autopause, height=4cm]{10}{vnlb3d_color/tennis_s40_pt4/tile040-150-070_}{001}{030}
			\begin{animateinline}[palindrome, autoplay]{10}
				\multiframe{30}{i=1+1}{%
				\begin{tabular}{c}
					\includegraphics[height=3.5cm]{data/derf/tennis/tile040-150-070_\zeropad{123}{\i}}$\,$ 
					\includegraphics[height=3.5cm]{vnlb3d_color/tennis_s40_pt4/nisy_tile040-150-070_\zeropad{123}{\i}}\\
					\includegraphics[height=3.5cm]{VBM4D/tennis_mp_s40/tile040-150-070_\zeropad{123}{\i}}$\,$ 
					\includegraphics[height=3.5cm]{vnlb3d_color/tennis_s40_pt4/tile040-150-070_\zeropad{123}{\i}}
				\end{tabular}
				}
			\end{animateinline}
	\end{center}

	\begin{center}
		\emph{Tennis} (detail). Top: original and noisy ($\sigma = 40$), bottom: V-BM4D-mp and VNLB.
	\end{center}
\end{frame}
\restoreframe

% \multipleframe
% % bus detail1 
% \begin{frame}{Results on color sequences}
% 	\begin{center}
% % 			\animategraphics[palindrome, controls, autopause, height=4cm]{10}{VBM4D/bus_mp_s10/tile060-020-020_}{001}{030}
% % 			\animategraphics[palindrome, controls, autopause, height=4cm]{10}{vnlb3d_color/bus_s10_pt4/tile060-020-020_}{001}{030}
% 			\begin{animateinline}[palindrome, autoplay]{10}
% 				\multiframe{30}{i=1+1}{%
% 				\begin{tabular}{c}
% 					\includegraphics[height=3.5cm]{data/derf/bus/tile060-020-020_\zeropad{123}{\i}}$\,$ 
% 					\includegraphics[height=3.5cm]{vnlb3d_color/bus_s10_pt4/nisy_tile060-020-020_\zeropad{123}{\i}}\\
% 					\includegraphics[height=3.5cm]{VBM4D/bus_mp_s10/tile060-020-020_\zeropad{123}{\i}}$\,$ 
% 					\includegraphics[height=3.5cm]{vnlb3d_color/bus_s10_pt4/tile060-020-020_\zeropad{123}{\i}}
% 				\end{tabular}
% 				}
% 			\end{animateinline}
% 	\end{center}
% 
% 	\begin{center}
% 		\emph{Bus} (detail). Top: original and noisy ($\sigma = 10$), bottom: V-BM4D-mp and VNLB.
% 	\end{center}
% \end{frame}
% \begin{frame}{Results on color sequences}
% 	\begin{center}
% % 			\animategraphics[palindrome, controls, autopause, height=4cm]{10}{VBM4D/bus_mp_s20/tile060-020-020_}{001}{030}
% % 			\animategraphics[palindrome, controls, autopause, height=4cm]{10}{vnlb3d_color/bus_s20_pt4/tile060-020-020_}{001}{030}
% 			\begin{animateinline}[palindrome, autoplay]{10}
% 				\multiframe{30}{i=1+1}{%
% 				\begin{tabular}{c}
% 					\includegraphics[height=3.5cm]{data/derf/bus/tile060-020-020_\zeropad{123}{\i}}$\,$ 
% 					\includegraphics[height=3.5cm]{vnlb3d_color/bus_s20_pt4/nisy_tile060-020-020_\zeropad{123}{\i}}\\
% 					\includegraphics[height=3.5cm]{VBM4D/bus_mp_s20/tile060-020-020_\zeropad{123}{\i}}$\,$ 
% 					\includegraphics[height=3.5cm]{vnlb3d_color/bus_s20_pt4/tile060-020-020_\zeropad{123}{\i}}
% 				\end{tabular}
% 				}
% 			\end{animateinline}
% 	\end{center}
% 
% 	\begin{center}
% 		\emph{Bus} (detail). Top: original and noisy ($\sigma = 20$), bottom: V-BM4D-mp and VNLB.
% 	\end{center}
% \end{frame}
% \begin{frame}{Results on color sequences}
% 	\begin{center}
% % 			\animategraphics[palindrome, controls, autopause, height=4cm]{10}{VBM4D/bus_mp_s40/tile060-020-020_}{001}{030}
% % 			\animategraphics[palindrome, controls, autopause, height=4cm]{10}{vnlb3d_color/bus_s40_pt4/tile060-020-020_}{001}{030}
% 			\begin{animateinline}[palindrome, autoplay]{10}
% 				\multiframe{30}{i=1+1}{%
% 				\begin{tabular}{c}
% 					\includegraphics[height=3.5cm]{data/derf/bus/tile060-020-020_\zeropad{123}{\i}}$\,$ 
% 					\includegraphics[height=3.5cm]{vnlb3d_color/bus_s40_pt4/nisy_tile060-020-020_\zeropad{123}{\i}}\\
% 					\includegraphics[height=3.5cm]{VBM4D/bus_mp_s40/tile060-020-020_\zeropad{123}{\i}}$\,$ 
% 					\includegraphics[height=3.5cm]{vnlb3d_color/bus_s40_pt4/tile060-020-020_\zeropad{123}{\i}}
% 				\end{tabular}
% 				}
% 			\end{animateinline}
% 	\end{center}
% 
% 	\begin{center}
% 		\emph{Bus} (detail). Top: original and noisy ($\sigma = 40$), bottom: V-BM4D-mp and VNLB.
% 	\end{center}
% \end{frame}
% 
% % bus detail2
% \begin{frame}{Results on color sequences}
% 	\begin{center}
% % 			\animategraphics[palindrome, controls, autopause, height=4cm]{10}{VBM4D/bus_mp_s10/tile110-150-010_}{001}{030}
% % 			\animategraphics[palindrome, controls, autopause, height=4cm]{10}{vnlb3d_color/bus_s10_pt4/tile110-150-010_}{001}{030}
% 			\begin{animateinline}[palindrome, autoplay]{10}
% 				\multiframe{30}{i=1+1}{%
% 				\begin{tabular}{c}
% 					\includegraphics[height=3.5cm]{data/derf/bus/tile110-150-010_\zeropad{123}{\i}}$\,$ 
% 					\includegraphics[height=3.5cm]{vnlb3d_color/bus_s10_pt4/nisy_tile110-150-010_\zeropad{123}{\i}}\\
% 					\includegraphics[height=3.5cm]{VBM4D/bus_mp_s10/tile110-150-010_\zeropad{123}{\i}}$\,$ 
% 					\includegraphics[height=3.5cm]{vnlb3d_color/bus_s10_pt4/tile110-150-010_\zeropad{123}{\i}}
% 				\end{tabular}
% 				}
% 			\end{animateinline}
% 	\end{center}
% 
% 	\begin{center}
% 		\emph{Bus} (detail). Top: original and noisy ($\sigma = 10$), bottom: V-BM4D-mp and VNLB.
% 	\end{center}
% \end{frame}
% \begin{frame}{Results on color sequences}
% 	\begin{center}
% % 			\animategraphics[palindrome, controls, autopause, height=4cm]{10}{VBM4D/bus_mp_s20/tile110-150-010_}{001}{030}
% % 			\animategraphics[palindrome, controls, autopause, height=4cm]{10}{vnlb3d_color/bus_s20_pt4/tile110-150-010_}{001}{030}
% 			\begin{animateinline}[palindrome, autoplay]{10}
% 				\multiframe{30}{i=1+1}{%
% 				\begin{tabular}{c}
% 					\includegraphics[height=3.5cm]{data/derf/bus/tile110-150-010_\zeropad{123}{\i}}$\,$ 
% 					\includegraphics[height=3.5cm]{vnlb3d_color/bus_s20_pt4/nisy_tile110-150-010_\zeropad{123}{\i}}\\
% 					\includegraphics[height=3.5cm]{VBM4D/bus_mp_s20/tile110-150-010_\zeropad{123}{\i}}$\,$ 
% 					\includegraphics[height=3.5cm]{vnlb3d_color/bus_s20_pt4/tile110-150-010_\zeropad{123}{\i}}
% 				\end{tabular}
% 				}
% 			\end{animateinline}
% 	\end{center}
% 
% 	\begin{center}
% 		\emph{Bus} (detail). Top: original and noisy ($\sigma = 20$), bottom: V-BM4D-mp and VNLB.
% 	\end{center}
% \end{frame}
% \begin{frame}{Results on color sequences}
% 	\begin{center}
% % 			\animategraphics[palindrome, controls, autopause, height=4cm]{10}{VBM4D/bus_mp_s40/tile110-150-010_}{001}{030}
% % 			\animategraphics[palindrome, controls, autopause, height=4cm]{10}{vnlb3d_color/bus_s40_pt4/tile110-150-010_}{001}{030}
% 			\begin{animateinline}[palindrome, autoplay]{10}
% 				\multiframe{30}{i=1+1}{%
% 				\begin{tabular}{c}
% 					\includegraphics[height=3.5cm]{data/derf/bus/tile110-150-010_\zeropad{123}{\i}}$\,$ 
% 					\includegraphics[height=3.5cm]{vnlb3d_color/bus_s40_pt4/nisy_tile110-150-010_\zeropad{123}{\i}}\\
% 					\includegraphics[height=3.5cm]{VBM4D/bus_mp_s40/tile110-150-010_\zeropad{123}{\i}}$\,$ 
% 					\includegraphics[height=3.5cm]{vnlb3d_color/bus_s40_pt4/tile110-150-010_\zeropad{123}{\i}}
% 				\end{tabular}
% 				}
% 			\end{animateinline}
% 	\end{center}
% 
% 	\begin{center}
% 		\emph{Bus} (detail). Top: original and noisy ($\sigma = 40$), bottom: V-BM4D-mp and VNLB.
% 	\end{center}
% \end{frame}
% \restoreframe

\begin{frame}{PSNRs for color test sequences}
	\begin{center}
		{\small
		\renewcommand{\tabcolsep}{2mm}
		\renewcommand{\arraystretch}{1.0}
		\begin{tabular}{ c | l |c c | c c | c c | c c}
			\hline
			\rule{0pt}{6pt}$\sigma$ & Method             & \multicolumn{2}{c}{Tennis}  & \multicolumn{2}{c}{Coastguard}&\multicolumn{2}{c}{Foreman}& \multicolumn{2}{c}{Bus}    \\\hline
			\multirow{5}{*}{$10$} & V-BM4D-tip           & \bsic{     } & \best{36.42} & \bsic{     } & \best{37.27} & \bsic{     } &       37.92  & \bsic{     } &       36.23  \\
			                      & V-BM4D-mp            & \bsic{35.64} &       35.90  & \bsic{36.06} &       36.30  & \bsic{36.96} &       37.21  & \bsic{35.10} &       35.38  \\
%			                      & V-BM4D-np            & \bsic{toto } &       35.56  & \bsic{toto } &       36.20  & \bsic{toto } &       36.90  & \bsic{toto } &       35.09  \\
			                      & V-BM3D               & \bsic{     } &       36.04  & \bsic{     } &       36.82  & \bsic{     } &       37.52  & \bsic{     } &       34.96  \\
%			                      & VNLB $n_t = 0$       & \bsic{toto } &       34.43  & \bsic{toto } &       36.70  & \bsic{toto } &       37.21  & \bsic{toto } &       35.58  \\
%			                      & VNLB $n_t = 3$       & \bsic{toto } &       35.04  & \bsic{toto } &       37.19  & \bsic{toto } &       37.88  & \bsic{toto } &       36.19  \\
			                      & VNLB $n_t = 5$       & \bsic{34.51} &       35.28  & \bsic{36.80} & \best{37.30} & \bsic{37.22} & \best{38.11} & \bsic{35.43} & \best{36.37} \\
%			                      & VNLB   $s_t = 1$     & \bsic{todo } &       todo   & \bsic{todo } &       todo   & \bsic{todo } &       todo   & \bsic{todo } &       todo   \\
			                      & VNLB   $s_t = 2$     & \bsic{35.65} &       36.19  & \bsic{37.42} &       38.05  & \Bsic{37.86} &       38.74  & \Bsic{35.80} &       36.63  \\
			                      & VNLB   $s_t = 3$     & \bsic{35.94} &       36.51  & \Bsic{37.57} &       38.35  & \Bsic{37.94} &       38.95  & \Bsic{35.86} &       36.79  \\
			                      & VNLB   $s_t = 4$     & \Bsic{36.02} & \Best{36.65} & \Bsic{37.53} & \Best{38.46} & \Bsic{37.85} & \Best{38.99} & \Bsic{35.86} & \Best{36.85} \\\hline
%
			\multirow{5}{*}{$20$} & V-BM4D-tip           & \bsic{     } & \best{32.88} & \bsic{     } & \best{33.61} & \bsic{     } & \best{34.62} & \bsic{     } & \best{32.27} \\
			                      & V-BM4D-mp            & \bsic{31.33} &       31.98  & \bsic{31.91} &       32.44  & \bsic{33.16} &       33.70  & \bsic{30.81} &       31.34  \\
%			                      & V-BM4D-np            & \bsic{toto } &       31.67  & \bsic{toto } &       32.24  & \bsic{toto } &       33.34  & \bsic{toto } &       30.95  \\
			                      & V-BM3D               & \bsic{     } &       32.54  & \bsic{     } &       33.39  & \bsic{     } &       34.49  & \bsic{     } &       31.03  \\
%			                      & VNLB $n_t = 0$       & \bsic{toto } &       30.65  & \bsic{toto } &       32.70  & \bsic{toto } &       33.64  & \bsic{toto } &       31.39  \\
%			                      & VNLB $n_t = 3$       & \bsic{toto } &       31.26  & \bsic{toto } &       33.35  & \bsic{toto } &       34.25  & \bsic{toto } &       32.11  \\
			                      & VNLB $n_t = 5$       & \bsic{30.76} &       31.49  & \bsic{32.94} & \best{33.55} & \bsic{33.69} &       34.45  & \bsic{31.58} & \best{32.35} \\
%			                      & VNLB   $s_t = 1$     & \bsic{todo } &       todo   & \bsic{todo } &       todo   & \bsic{todo } &       todo   & \bsic{todo } &       todo   \\
			                      & VNLB   $s_t = 2$     & \bsic{32.12} &       32.56  & \bsic{33.98} &       34.30  & \bsic{34.55} &       35.32  & \bsic{32.11} &       32.71  \\
			                      & VNLB   $s_t = 3$     & \bsic{32.52} &       33.00  & \Bsic{34.15} &       34.60  & \Bsic{34.71} &       35.64  & \Bsic{32.26} &       32.97  \\
			                      & VNLB   $s_t = 4$     & \Bsic{32.67} & \Best{33.22} & \Bsic{34.15} & \Best{34.78} & \Bsic{34.63} & \Best{35.75} & \Bsic{32.30} & \Best{33.12} \\\hline
%
			\multirow{5}{*}{$40$} & V-BM4D-tip           & \bsic{     } & \best{29.52} & \bsic{     } & \best{30.00} & \bsic{     } & \best{31.30} & \bsic{     } &       28.32  \\
			                      & V-BM4D-mp            & \bsic{27.34} &       28.14  & \bsic{27.81} &       28.73  & \bsic{29.01} &       30.09  & \bsic{26.67} &       27.44  \\
%			                      & V-BM4D-np            & \bsic{toto } &       27.97  & \bsic{toto } &       28.43  & \bsic{toto } &       29.69  & \bsic{toto } &       27.02  \\
			                      & V-BM3D               & \bsic{     } &       29.20  & \bsic{     } & \best{29.99} & \bsic{     } &       31.17  & \bsic{     } &       27.34  \\
%			                      & VNLB $n_t = 0$       & \bsic{toto } &       27.43  & \bsic{toto } &       28.88  & \bsic{toto } &       30.15  & \bsic{toto } &       27.60  \\
%			                      & VNLB $n_t = 3$       & \bsic{toto } &       27.97  & \bsic{toto } &       29.60  & \bsic{toto } &       30.72  & \bsic{toto } &       28.23  \\
			                      & VNLB $n_t = 5$       & \bsic{27.17} &       28.11  & \bsic{28.81} &       29.78  & \bsic{29.72} &       30.82  & \bsic{27.65} & \best{28.40} \\
%			                      & VNLB   $s_t = 1$     & \bsic{todo } &       todo   & \bsic{todo } &       todo   & \bsic{todo } &       todo   & \bsic{todo } &       todo   \\
			                      & VNLB   $s_t = 2$     & \bsic{28.44} &       29.28  & \bsic{30.18} &       30.94  & \bsic{30.87} &       32.00  & \bsic{28.21} &       28.96  \\
			                      & VNLB   $s_t = 3$     & \Bsic{28.86} &       29.73  & \Bsic{30.43} &       31.23  & \Bsic{31.03} &       32.39  & \Bsic{28.32} &       29.23  \\
			                      & VNLB   $s_t = 4$     & \Bsic{28.99} & \Best{29.96} & \Bsic{30.40} & \Best{31.34} & \Bsic{30.95} & \Best{32.55} & \Bsic{28.31} & \Best{29.39} \\\hline
		\end{tabular}}
% Command to print rounded psnrs
% for i in $(cat bus_s40_pt*/measures | grep PSNR_final | sed "s/^-PSNR_final\ =\ "//); do  echo "scale=2;(((10^2)*$i)+0.5)/(10^2)" | bc; done

		\bigskip

		PSNRs obtained for the four classic color test sequences.
	\end{center}
\end{frame}

\begin{frame}{Temporal consistency of the result}
	\begin{center}
		\includegraphics<1>[width=.9\textwidth]{figures/bus_slice100_s40.png}
		\includegraphics<2>[width=.9\textwidth]{figures/bus_slice100_s40_vbm4d.png}
		\includegraphics<3>[width=.9\textwidth]{figures/bus_slice100_s40_vnlb3d.png}
		\includegraphics<4>[width=.9\textwidth]{figures/bus_slice100.png}
	%	\includegraphics<2>[width=.7\textwidth]{figures/bus_slice100_s10.png}
	%	\includegraphics<3>[width=.7\textwidth]{figures/bus_slice100_s10_vbm4d.png}
	%	\includegraphics<4>[width=.7\textwidth]{figures/bus_slice100_s10_vnlb3d.png}
	%	\includegraphics<5>[width=.7\textwidth]{figures/bus_slice100_s20.png}
	%	\includegraphics<6>[width=.7\textwidth]{figures/bus_slice100_s20_vbm4d.png}
	%	\includegraphics<7>[width=.7\textwidth]{figures/bus_slice100_s20_vnlb3d.png}
	\end{center}

	\bigskip

	\bigskip

	\centerline{Horizontal slice of a video. Each row corresponds to a different frame.}
		\begin{overprint}
			\onslide<1> \centerline{Original video contaminted with AWGN of $\sigma = 40$.}
			\onslide<2> \centerline{Result of V-BM4D-mp.}
			\onslide<3> \centerline{Result of VNLB.}
			\onslide<4> \centerline{Original video.}
		\end{overprint}
		
\end{frame}

\multipleframe
% stefan detail1 
\begin{frame}{Example of a challenging sequence}
	\begin{center}
% 			\animategraphics[palindrome, controls, autopause, height=4cm]{2}{VBM4D/stefan_mono_mp_s20/tile090-180-265_}{001}{010}
% 			\animategraphics[palindrome, controls, autopause, height=4cm]{2}{vnlb3d_mono/stefan_mono_s20_pt4/tile090-180-265_}{001}{010}
			\begin{animateinline}[palindrome, autoplay]{10}
				\multiframe{30}{i=1+1}{%
				\begin{tabular}{c}
					\includegraphics[height=3.5cm]{data/derf/stefan_mono/tile090-180-265_\zeropad{123}{\i}}$\,$ 
					\includegraphics[height=3.5cm]{vnlb3d_mono/stefan_mono_s10_pt4/nisy_tile090-180-265_\zeropad{123}{\i}}\\
					\includegraphics[height=3.5cm]{VBM4D/stefan_mono_mp_s10/tile090-180-265_\zeropad{123}{\i}}$\,$ 
					\includegraphics[height=3.5cm]{vnlb3d_mono/stefan_mono_s10_pt4/tile090-180-265_\zeropad{123}{\i}}
				\end{tabular}
				}
			\end{animateinline}
	\end{center}

	\begin{center}
		\emph{Stefan} (detail). Top: original and noisy ($\sigma = 10$), bottom: V-BM4D-mp and VNLB.
	\end{center}
\end{frame}
\begin{frame}{Example of a challenging sequence}
	\begin{center}
% 			\animategraphics[palindrome, controls, autopause, height=4cm]{2}{VBM4D/stefan_mono_mp_s20/tile090-180-265_}{001}{010}
% 			\animategraphics[palindrome, controls, autopause, height=4cm]{2}{vnlb3d_mono/stefan_mono_s20_pt4/tile090-180-265_}{001}{010}
			\begin{animateinline}[palindrome, autoplay]{10}
				\multiframe{30}{i=1+1}{%
				\begin{tabular}{c}
					\includegraphics[height=3.5cm]{data/derf/stefan_mono/tile090-180-265_\zeropad{123}{\i}}$\,$ 
					\includegraphics[height=3.5cm]{vnlb3d_mono/stefan_mono_s20_pt4/nisy_tile090-180-265_\zeropad{123}{\i}}\\
					\includegraphics[height=3.5cm]{VBM4D/stefan_mono_mp_s20/tile090-180-265_\zeropad{123}{\i}}$\,$ 
					\includegraphics[height=3.5cm]{vnlb3d_mono/stefan_mono_s20_pt4/tile090-180-265_\zeropad{123}{\i}}
				\end{tabular}
				}
			\end{animateinline}
	\end{center}

	\begin{center}
		\emph{Stefan} (detail). Top: original and noisy ($\sigma = 20$), bottom: V-BM4D-mp and VNLB.
	\end{center}
\end{frame}
\begin{frame}{Example of a challenging sequence}
	\begin{center}
% 			\animategraphics[palindrome, controls, autopause, height=4cm]{2}{VBM4D/stefan_mono_mp_s20/tile090-180-265_}{001}{010}
% 			\animategraphics[palindrome, controls, autopause, height=4cm]{2}{vnlb3d_mono/stefan_mono_s20_pt4/tile090-180-265_}{001}{010}
			\begin{animateinline}[palindrome, autoplay]{10}
				\multiframe{30}{i=1+1}{%
				\begin{tabular}{c}
					\includegraphics[height=3.5cm]{data/derf/stefan_mono/tile090-180-265_\zeropad{123}{\i}}$\,$ 
					\includegraphics[height=3.5cm]{vnlb3d_mono/stefan_mono_s40_pt4/nisy_tile090-180-265_\zeropad{123}{\i}}\\
					\includegraphics[height=3.5cm]{VBM4D/stefan_mono_mp_s40/tile090-180-265_\zeropad{123}{\i}}$\,$ 
					\includegraphics[height=3.5cm]{vnlb3d_mono/stefan_mono_s40_pt4/tile090-180-265_\zeropad{123}{\i}}
				\end{tabular}
				}
			\end{animateinline}
	\end{center}

	\begin{center}
		\emph{Stefan} (detail). Top: original and noisy ($\sigma = 40$), bottom: V-BM4D-mp and VNLB.
	\end{center}
\end{frame}
\restoreframe

% stefan detail2 
\multipleframe
\begin{frame}{Example of a challenging sequence}
	\begin{center}
% 			\animategraphics[palindrome, controls, autopause, height=4cm]{2}{VBM4D/stefan_mono_mp_s20/tile090-180-265_}{001}{010}
% 			\animategraphics[palindrome, controls, autopause, height=4cm]{2}{vnlb3d_mono/stefan_mono_s20_pt4/tile090-180-265_}{001}{010}
			\begin{animateinline}[palindrome, autoplay]{10}
				\multiframe{30}{i=1+1}{%
				\begin{tabular}{c}
					\includegraphics[height=3.5cm]{data/derf/stefan_mono/tile060-130-005_\zeropad{123}{\i}}$\,$ 
					\includegraphics[height=3.5cm]{vnlb3d_mono/stefan_mono_s10_pt4/nisy_tile060-130-005_\zeropad{123}{\i}}\\
					\includegraphics[height=3.5cm]{VBM4D/stefan_mono_mp_s10/tile060-130-005_\zeropad{123}{\i}}$\,$ 
					\includegraphics[height=3.5cm]{vnlb3d_mono/stefan_mono_s10_pt4/tile060-130-005_\zeropad{123}{\i}}
				\end{tabular}
				}
			\end{animateinline}
	\end{center}

	\begin{center}
		\emph{Stefan} (detail). Top: original and noisy ($\sigma = 10$), bottom: V-BM4D-mp and VNLB.
	\end{center}
\end{frame}
\begin{frame}{Example of a challenging sequence}
	\begin{center}
% 			\animategraphics[palindrome, controls, autopause, height=4cm]{2}{VBM4D/stefan_mono_mp_s20/tile090-180-265_}{001}{010}
% 			\animategraphics[palindrome, controls, autopause, height=4cm]{2}{vnlb3d_mono/stefan_mono_s20_pt4/tile090-180-265_}{001}{010}
			\begin{animateinline}[palindrome, autoplay]{10}
				\multiframe{30}{i=1+1}{%
				\begin{tabular}{c}
					\includegraphics[height=3.5cm]{data/derf/stefan_mono/tile060-130-005_\zeropad{123}{\i}}$\,$ 
					\includegraphics[height=3.5cm]{vnlb3d_mono/stefan_mono_s20_pt4/nisy_tile060-130-005_\zeropad{123}{\i}}\\
					\includegraphics[height=3.5cm]{VBM4D/stefan_mono_mp_s20/tile060-130-005_\zeropad{123}{\i}}$\,$ 
					\includegraphics[height=3.5cm]{vnlb3d_mono/stefan_mono_s20_pt4/tile060-130-005_\zeropad{123}{\i}}
				\end{tabular}
				}
			\end{animateinline}
	\end{center}

	\begin{center}
		\emph{Stefan} (detail). Top: original and noisy ($\sigma = 20$), bottom: V-BM4D-mp and VNLB.
	\end{center}
\end{frame}
\begin{frame}{Example of a challenging sequence}
	\begin{center}
% 			\animategraphics[palindrome, controls, autopause, height=4cm]{2}{VBM4D/stefan_mono_mp_s20/tile090-180-265_}{001}{010}
% 			\animategraphics[palindrome, controls, autopause, height=4cm]{2}{vnlb3d_mono/stefan_mono_s20_pt4/tile090-180-265_}{001}{010}
			\begin{animateinline}[palindrome, autoplay]{10}
				\multiframe{30}{i=1+1}{%
				\begin{tabular}{c}
					\includegraphics[height=3.5cm]{data/derf/stefan_mono/tile060-130-005_\zeropad{123}{\i}}$\,$ 
					\includegraphics[height=3.5cm]{vnlb3d_mono/stefan_mono_s40_pt4/nisy_tile060-130-005_\zeropad{123}{\i}}\\
					\includegraphics[height=3.5cm]{VBM4D/stefan_mono_mp_s40/tile060-130-005_\zeropad{123}{\i}}$\,$ 
					\includegraphics[height=3.5cm]{vnlb3d_mono/stefan_mono_s40_pt4/tile060-130-005_\zeropad{123}{\i}}
				\end{tabular}
				}
			\end{animateinline}
	\end{center}

	\begin{center}
		\emph{Stefan} (detail). Top: original and noisy ($\sigma = 40$), bottom: V-BM4D-mp and VNLB.
	\end{center}
\end{frame}
\restoreframe


% \begin{frame}{Computation time}
% 
% 	Computations performed in \texttt{boucantrin} server, using 8 CPU cores.\\
% 	For CIF (352x288) RGB videos, with $\sigma = 40$: 
% 
% 	\bigskip
% 
% 	\begin{center}
% 	\begin{tabular}{r | c c c c}
% 		$p_t$ & 1 & 2 & 3 & 4 \\
% 		$s/\textnormal{frame}$ & x & 7 & 15 & 30 \\
% 	\end{tabular}
% 	\end{center}
% 
% 	\vspace{2cm}
% 
% 	For example, with $p_t = 4$, it takes 2.5 hours ($\times 8$ cores) for 300 CIF frames!
% 
% \end{frame}
% 
% 

\begin{frame}{Conclusions}

	\setbeamertemplate{itemize items}[triangle]

	\begin{itemize}\itemsep=.5cm
		\item Video denoising by Bayesian filtering of groups of similar spatio-temporal patches.
		\item Temporal consistency without motion estimation.
		\item The proposed model works well on objects with uniform motion.
	%	\item Problems:
	%	\begin{itemize}\itemsep=.4cm
				\item Adherence problem: background texture follows the motion of the foreground object.
				\item Fast changes in motion create artifacts (particularly for
					higher noise levels). 
				\item High computational cost: we are currently searching for stategies to alleviate this.
	%		\end{itemize}
		\item Future work: estimate motion from the basic estimate, to guide the search of similar patches 
			in the final estimate.
	\end{itemize}


\end{frame}

\end{document}

