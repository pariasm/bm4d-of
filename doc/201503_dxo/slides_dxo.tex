% \documentclass{beamer}
\documentclass[mathserif]{beamer}

\mode<presentation> {
% \usetheme{Madrid}
%\usetheme{default}
\usetheme{Boadilla}
% \usecolortheme{beaver}
% \usecolortheme{rose}
}
% removes navigation symbols
\setbeamertemplate{navigation symbols}{}

\usepackage[english,activeacute]{babel}
\usepackage[english]{layout}
\usepackage{amsfonts,amsmath,amssymb,amsthm}
\usepackage{pgf}
%\usepackage{movie15}
\usepackage{graphicx}
\usepackage{rotating}
\usepackage{color}
\usepackage[utf8]{inputenc}
\usepackage[multidot]{grffile}
\usepackage{ucs}
% \usepackage[T1]{fontenc}
%\usepackage{subfigure}

% from rida's preamble
\usepackage{epsfig,xspace}
\usepackage{setspace}
\usepackage{threeparttable}
\usepackage{subfloat}
\usepackage{pstricks,pst-node}
\usepackage[normal,tight,center]{subfigure}
\usepackage{pifont}
\usepackage{multimedia}
%\usepackage{media9}
%\usepackage{enumitem}
\usepackage{animate}
\usepackage{caption}
\usepackage{multirow}


% the following is used to bind frames with the same frame number
\usepackage{etoolbox}

\newcounter{multipleslide}

\makeatletter%
\newcommand{\multipleframe}{%
\setcounter{multipleslide}{\value{framenumber}}
\stepcounter{multipleslide}
\patchcmd{\beamer@@tmpl@footline}% <cmd>
	{\insertframenumber}% <search>
	{\themultipleslide}% <replace>
	{}% <success>
	{}% <failure>
}
\newcommand{\restoreframe}{%
\patchcmd{\beamer@@tmpl@footline}% <cmd>
	{\themultipleslide}% <search>
	{\insertframenumber}% <replace>
	{}% <success>
	{}% <failure>
\setcounter{framenumber}{\value{multipleslide}}%
}
\makeatother%




%%%%%%%%%%%%%%%%%%%%%%%%%%%%%%%%%%%%%%%
% Running 'pdflatex --shell-escape...' will automatically recognize
% the eps files in the includegraphics and convert them to pdf.
% No file rename nor change to the document is needed.
%%%%%%%%%%%%%%%%%%%%%%%%%%%%%%%%%%%%%%%
\usepackage{ifpdf}
\ifpdf
\DeclareGraphicsRule{.eps}{pdf}{.pdf}{`epstopdf #1}
\usepackage{epstopdf}
\epstopdfsetup{suffix=-\SourceExt-converted-to}
\pdfcompresslevel=0   %\pdfcompresslevel=9
\pdfcompresslevel0
\fi


% para poder incluir figuras .pdftex del xfig
\DeclareGraphicsRule{.pdftex}{pdf}{.pdftex}{}

\newcommand{\nada}[1]   {}

\newcommand{\ma}[1]{\boldsymbol{#1}}
\newcommand{\ie}{\textit{i.e.} }
\newcommand{\eg}{\textit{e.g.} }
\newcommand{\etal}{\textit{et al}. }
\newcommand{\tras}[1]{#1^{\mathrm{T}}}
\newcommand{\herm}[1]{#1^{\mathrm{H}}}
\newcommand{\con}[1]{#1^{\mathrm{*}}}
\newcommand{\E}{\mathbb{E}}
\newcommand{\tech}[1]{\overline{#1}}
\newcommand{\nspace}{\!\!\!\!}
\newcommand{\nmbr}[1]{\oldstylenums{#1}}

\newcommand{\eps}{\varepsilon}
\newcommand{\R}{{\mathbb R}}
\newcommand{\Q}{{\mathbb Q}}
\newcommand{\N}{{\mathbb N}}
\newcommand{\Z}{{\mathbb Z}}
\newcommand{\C}{{\mathcal C}}
\renewcommand{\H}{{\mathcal H}}
\newcommand{\F}{{\mathcal F}}

\DeclareMathOperator*{\argmin}{arg\,min}
\DeclareMathOperator*{\argmax}{arg\,max}
\DeclareMathOperator*{\median}{median}



\definecolor{MyOrange}  {cmyk}{0,0.73,1.0,0}
\definecolor{MyGreen}   {cmyk}{1,0,1,0.0}

% Rida's colors and commands
\definecolor{Red}{rgb}{0.9,0.0,0.1}
\definecolor{Brown}{rgb}{0.55,0.27,0.1}
\definecolor{Brownie}{rgb}{0.75,0.27,0.1}
\definecolor{Yellow}{rgb}{1,1,0}
\definecolor{White}{rgb}{1,1,1}
\definecolor{formule}{rgb}{0.75,0.27,0.1}
\definecolor{formule}{rgb}{.7,1,0}
\definecolor{violet}{rgb}{0.7,0,.9}
\definecolor{darkgreen}{rgb}{0,.7,0}

%\newcommand{\reference}[1] {{\scriptsize \color{gray}  #1 }} 
\newcommand{\reference}[1] {{\color{gray}  #1 }} 
\newcommand{\rmk}[1]{{\color{Red} #1}}

%\theoremstyle{plain}\newtheorem{theorem}{Theorem}[chapter]
%\theoremstyle{plain}\newtheorem{proposition}{Proposition}[chapter]
%\theoremstyle{plain}\newtheorem{lemma}{Lemma}[chapter]
%\theoremstyle{definition}\newtheorem{definition}{Definition}[chapter]

% New definition of square root:
% it renames \sqrt as \oldsqrt
\let\oldsqrt\sqrt
% it defines the new \sqrt in terms of the old one
\def\sqrt{\mathpalette\DHLhksqrt}
\def\DHLhksqrt#1#2{%
\setbox0=\hbox{$#1\oldsqrt{#2\,}$}\dimen0=\ht0
\advance\dimen0-0.2\ht0
\setbox2=\hbox{\vrule height\ht0 depth -\dimen0}%
{\box0\lower0.4pt\box2}}


%gets rid of bottom navigation bars
% \setbeamertemplate{footline}[frame number]{}

%gets rid of navigation symbols
\setbeamertemplate{navigation symbols}{}

\title[Video denoising]{$\,$ \\ $\,$ \\ Preliminary results on patch-based video denoising}
\author[Pablo Arias]{Pablo Arias\\ Jean-Michel Morel\vspace{.5cm}\\ $\,$ \\ $\,$ \\ $\,$ \\
Meeting CMLA-DxO \\March 17, 2015}
\institute[CMLA]{}
\date[]{}

%\author[Pablo Arias]{Pablo Arias$^*$, Vicent Caselles$^*$, Gabriele Facciolo$^\dagger$ \\ Rida Sadek$^*$, Guillermo Sapiro$^\ddagger$ \\ $\,$ \\  $\,$ \\ $\,$ \\ $\,$ \\
%\institute[UPF]{* Universitat Pompeu Fabra\\ $\dagger$ ENS Cachan \\ $\ddagger$ Duke University}

% \AtBeginSection[]  % "Beamer, do the following at the start of every section"
% {\begin{frame}<beamer>
% \frametitle{Outline} % make a frame titled "Outline"
% \tableofcontents[currentsection]  % show TOC and highlight current section
% \end{frame}}
%
% \AtBeginSubsection[]  % "Beamer, do the following at the start of every section"
% {\begin{frame}<beamer>
% \frametitle{Outline} % make a frame titled "Outline"
% \tableofcontents[currentsection,currentsubsection]  % show TOC and highlight current section
% \end{frame}}

\graphicspath{{./figures/}}

\begin{document}

\begin{frame}
    \titlepage
\end{frame}

% \begin{frame}{Gradient-domain video editing}
% 	\begin{center}
% 		\includegraphics[width=0.19\textwidth]{figures_rida/vid_edit/screen_seq_screen_025_cropped}
% 		\includegraphics[width=0.19\textwidth]{figures_rida/vid_edit/screen_seq_mask_red_031_cropped}
% 		\includegraphics[width=0.19\textwidth]{figures_rida/vid_edit/screen_seq_mask_red_037_cropped}
% 		\includegraphics[width=0.19\textwidth]{figures_rida/vid_edit/screen_seq_mask_red_041_cropped}
% 		\includegraphics[width=0.19\textwidth]{figures_rida/vid_edit/screen_seq_screen_045_cropped}	 
% 		\\
% 		\includegraphics[width=0.19\textwidth]{figures_rida/vid_edit/screen_seq_screen_025_cropped}
% 		\includegraphics[width=0.19\textwidth]{figures_rida/vid_edit/screen_seq_screen_031_cropped}
% 		\includegraphics[width=0.19\textwidth]{figures_rida/vid_edit/screen_seq_screen_037_cropped}
% 		\includegraphics[width=0.19\textwidth]{figures_rida/vid_edit/screen_seq_screen_041_cropped}
% 		\includegraphics[width=0.19\textwidth]{figures_rida/vid_edit/screen_seq_screen_045_cropped}\\
% 	\end{center}
% \end{frame}

\section{Introduction}


\begin{frame}{Introduction}

	We are exploring several patch-based approaches for video denoising.
	We have currently extended to video:

	\vspace{.2cm}

	\begin{enumerate}
		\item[1.] non-local Bayes (based on IPOL C++ code of Marc Lebrun)
		\item[2.] a two-stage version of non-local means (based on same code)
	\end{enumerate}

	\vspace{1cm}

	IPOL demos for both methods, together with:

	\vspace{.2cm}

	\begin{enumerate}
		\item[1.] Video BM3D (Matlab/C code by Dabov, Danieyan, Foi; only grayscale)
		\item[2.] BM4D (Matlab/C code by Maggioni, Foi)
	\end{enumerate}
	
\end{frame}

\begin{frame}{Video NL-Bayes}

	Straightforward adaptation of non-local Bayes to video: patches are
	searched for in a spatio-temporal search window, of size $W_x \times W_x
	\times W_t$. 
	
	\pause

	\vspace{.6cm}

	Parameters based on default parameters for NL-Bayes for images:
	\vspace{.2cm}
	\begin{itemize} \itemsep=.2cm
		\item $W_t$ selected by user
		\item The number of similar patches $N_{\text{sim}}$ used for the
			Bayesian estimation step: 
			\[N_{\text{sim}} = N_{\text{sim},0} W_t,\]
			where $N_{\text{sim},0}$ default parameter for image NL-Bayes.
		\item Rest of parameters are the default ones from image NL-Bayes.
	\end{itemize}

	\vspace{1cm}

\end{frame}

\begin{frame}{Video two-stage NL-Means}

	As in Video NL-Bayes, we build groups of similar patches which are jointly
	denoised. 

	\vspace{.5cm}
	
	Instead of performing the Bayesian group estimation, we estimate each patch
	in the group by \structure{averages weighted by patch similarity} of the
	group patches, as in NL-means.
	
	\vspace{.5cm}
	
	Patches in group are then \structure{aggregated to form an image}.

	\vspace{.5cm}
	
	Two stages: basic and final estimates. In the final stage, the basic
	estimate is used to build the groups of similar patches and to compute the
	similarity weights.

\end{frame}

\multipleframe
\begin{frame}{Video two-stage NL-Means}
	
	\structure{Basic estimate:} For each $x$:

	\begin{enumerate}
		\item[1.] Build a group $\mathcal G_x$ of patches similar to $\hat p_x$ ($L^2$
			distance between $Y$ component).
	
		\item[2.] For each noisy patch $\hat p_y$ in the group $\mathcal G_x$:

	\[p_y^{\text{NLM},b} = \frac1{C}\sum_{\hat p_z\in\mathcal G_x} w(\hat p_y, \hat p_z)\hat p_z.\]

		Weights computed following [Buades, Coll, Morel; IPOL'13]:

	\[\quad\displaystyle w(\hat p_y, \hat p_z) = \exp\left(-\frac1{h_b^2}\max(\|\hat p_y - \hat p_z\|_Y^2, 2\sigma^2/3)\right)\]
	
		\item[3.] Aggregate the estimated patches $p_y^{\text{NLM},b}$ on an image $u_b$.

	\end{enumerate}
	
\end{frame}

\begin{frame}{Video two-stage NL-Means}
	
	\structure{Final estimate:} For each $x$:

	\begin{enumerate}
		\item[1.] Build a group $\mathcal G_x$ of noisy patches such that their basic
			estimate is similar to $p_x^b$ ($L^2$ distance
			between color patches of $y_b)$.
	
		\item[2.] For each noisy patch $\hat p_y$ in the group $\mathcal G^n_x$:

	\[p_y^{\text{NLM},f} = \frac1{C}\sum_{\hat p_z\in\mathcal G_x} w(p^b_y, p^b_z)\hat p_z.\]

		Weights computed following [Buades, Coll, Morel; IPOL'13]:

	\[w(p^b_y, p^b_z) = \exp\left(-\frac1{h_f^2}\max(\|p^b_y - p^b_z\|_{\text{RGB}}^2, 2\gamma\sigma^2)\right).\]
	
		\item[3.] Aggregate the estimated patches $p_y^{\text{NLM},f}$ on an image $u$.

	\end{enumerate}
	
\end{frame}
\restoreframe

\begin{frame}{Video two-stage NL-Means}

	\begin{itemize} \itemsep=.4cm
		\item $W_t$ selected by user.
		\item Rest of patch search parameters based on default parameters of
			Video NL-Bayes.
		\item Parameters for basic NL-means estimate based on default parameters
			in [Buades, Coll, Morel; IPOL'13]. 
		\item Parameters for final NL-means estimate set by trial-and-error based
			on a few results.
	\end{itemize}

	\vspace{1cm}

\end{frame}

\multipleframe
\begin{frame}{Video BM3D}

	As in Video NL-Bayes, groups of similar patches are jointly denoised. Groups
	are considered 3D signals, which are filtered in a 3D separable transformed
	domain.

	\vspace{.7cm}
	
	\structure{Basic estimate:} For each $x$:

	\begin{enumerate}\itemsep=.5cm
		\item[1.] Build a 3D group $\mathcal G_x$ of noisy patches similar to
			$\hat p_x$ (using $L^2$ distance between $Y$ component and predictive
			search).
	
		\item[2.] Hard thresholding on separable transform domain (separable
			bi-orthogonal wavelet on spatial axes + Haar wavelet
			on group axis).

		\item[3.] Aggregate the estimated patches on an image $u_b$, the
			basic estimate.

	\end{enumerate}
	
\end{frame}

\begin{frame}{Video BM3D}

	As in Video NL-Bayes, groups of similar patches are jointly denoised. Groups
	are considered 3D signals, which are filtered in a 3D separable transformed
	domain.
	
	\vspace{.7cm}
	
	\structure{Final estimate:} For each $x$:

	\begin{enumerate}\itemsep=.5cm
		\item[1.] Build 3D groups $\mathcal G_x$ of noisy patches \structure{and
			basic patches} similar to $p^b_x$ (using $L^2$ distance between $Y$
			component \structure{of basic estimate} and predictive search).
	
		\item[2.] \structure{Wiener filtering} on separable transform domain (\structure{2D DTC} on spatial
			axes + Haar wavelet on group axis). \structure{Coefficients of Wiener filter are estimated
			from the transformed basic group.}

		\item[3.] Aggregate the estimated patches on an image $u_f$, the
			\structure{final estimate}.

	\end{enumerate}
	
\end{frame}
\restoreframe


\multipleframe
\begin{frame}{Video BM4D}

	Collaborative filtering of \structure{spatio-temporal 3D patches}: a patch
	that spans several frames. In each frame, the top-left corner follows a
	motion trajectory.


	\vspace{.6cm}
	
	\structure{Basic estimate:}
	\begin{enumerate} \itemsep=.5cm
		\item[1.] For each $x$ build a trajectory $\text{Traj}(x)$ of a variable
			number of frames (block matching in noisy video $+$ temporal smoothing).
		\item[2.] For each $x$, let $p_x$ be the spatio-temporal patch at $x$:
			\begin{enumerate} \itemsep=.2cm
				\item[2.1.] Build a 4D group $\mathcal G_x$ by stacking 3D noisy
					spatio-temporal patches similar to $p_x$ ($L^2$ distance between
					$Y$ component).
				\item[2.2.] Thresholding in transform domain of 4D group
					using separable 4D transform. Different basis for different
					spatial, temporal and ``group'' dimensions.
				\item[2.3.] Weighted aggregation based on 4D transform sparsity.
			\end{enumerate}
	\end{enumerate}

\end{frame}

\begin{frame}{Video BM4D}

	Collaborative filtering of \structure{spatio-temporal 3D patches}: a patch
	that spans several frames. In each frame, the top-left corner follows a
	motion trajectory.


	\vspace{.6cm}
	
	\structure{Final estimate:}
	\begin{enumerate} \itemsep=.5cm
		\item[1.] For each $x$ build a trajectory $\text{Traj}(x)$ of a variable
			number of frames (block matching in \structure{basic estimate} $+$ temporal smoothing).
		\item[2.] For each $x$, let $p_x$ be the spatio-temporal patch at $x$:
			\begin{enumerate} \itemsep=.2cm
				\item[2.1.] Build a 4D group $\mathcal G_x$ by stacking 3D noisy
					spatio-temporal patches similar to $p_x$ ($L^2$ distance in
					\structure{basic estimate}).
				\item[2.2.] \structure{Wiener filtering} in transform domain of 4D
					group using separable 4D transform. Different basis for
					different spatial, temporal and ``group'' dimensions.
				\item[2.3.] Weighted aggregation based on 4D \structure{filtered coefficients}.
			\end{enumerate}
	\end{enumerate}

\end{frame}
\restoreframe

\begin{frame}[fragile]{IPOL demos and results}
	Demos on:\\
	\verb+http://dev.ipol.im/~pariasm/ipol_demo/video_nlbayes/+\\
	\verb+http://dev.ipol.im/~pariasm/ipol_demo/vbm3d/+\\
	\verb+http://dev.ipol.im/~pariasm/ipol_demo/vbm4d/+

	\vspace{2cm}

	Show results\dots

\end{frame}

\multipleframe
\begin{frame}{PSNRs of basic estimates}

	{\small
	\begin{tabular}{| c | c |c c c c c|}
		\hline \hline
		$\sigma$  & Method \textbackslash Video & Army & DogDance & Evergreen & Mequon & Walking \\\hline\hline
		\multirow{5}{*}{$10$} & BM4D           & \textbf{37.48} & \textbf{35.60} & \textbf{35.29} & \textbf{37.89} & \textbf{38.67} \\%\cline{2-7}
		                      & VNLB $W_t = 0$ &         34.17  &         33.31  &         32.99  &         34.76  &         34.82  \\%\cline{2-7}
		                      & VNLB $W_t = 1$ &         36.16  &         34.67  &         34.26  &         37.09  &         37.22  \\%\cline{2-7}
		                      & VNLB $W_t = 2$ &         36.46  &         34.82  &         34.40  &         37.33  &         37.58  \\\hline
%									 
		\multirow{5}{*}{$25$} & BM4D           & \textbf{32.24} & \textbf{31.07} & \textbf{30.48} &         32.89  & \textbf{33.54} \\%\cline{2-7}
		                      & VNLB $W_t = 0$ &         30.10  &         29.47  &         29.02  &         30.88  &         30.95  \\%\cline{2-7}
		                      & VNLB $W_t = 1$ &         31.89  &         30.80  &         30.28  &         32.84  &         32.96  \\%\cline{2-7}
		                      & VNLB $W_t = 2$ &         32.18  &         30.97  &         30.45  & \textbf{33.10} &         33.26  \\\hline
%									 
		\multirow{5}{*}{$40$} & BM4D           & \textbf{29.53} & \textbf{28.59} &         27.93  &         29.94  & \textbf{30.55} \\%\cline{2-7}
		                      & VNLB $W_t = 0$ &         26.99  &         26.58  &         26.23  &         27.56  &         27.59  \\%\cline{2-7}
		                      & VNLB $W_t = 1$ &         29.10  &         28.25  &         27.82  &         29.77  &         29.85  \\%\cline{2-7}
									 & VNLB $W_t = 2$ &         29.46  &         28.50  & \textbf{28.07} & \textbf{30.13} &         30.25  \\\hline\hline
	\end{tabular}}

\end{frame}

\begin{frame}{PSNRs of final estimates}

	{\small
	\begin{tabular}{| c | c |c c c c c|}
		\hline \hline
		$\sigma$  & Method \textbackslash Video & Army & DogDance & Evergreen & Mequon & Walking \\\hline\hline
		\multirow{5}{*}{$10$} & BM4D           & \textbf{37.77} & \textbf{35.70} & \textbf{35.40} &         38.09  &         38.85  \\%\cline{2-7}
		                      & VNLB $W_t = 0$ &         36.38  &         34.88  &         34.27  &         37.90  &         38.10  \\%\cline{2-7}
		                      & VNLB $W_t = 1$ &         37.17  &         35.31  &         34.80  & \textbf{38.79} & \textbf{39.16} \\%\cline{2-7}
		                      & VNLB $W_t = 2$ &         37.18  &         35.27  &         34.77  &         38.64  &         39.10  \\\hline
%									 
		\multirow{5}{*}{$25$} & BM4D           &         32.91  & \textbf{31.50} & \textbf{30.94} &         33.60  &         34.27  \\%\cline{2-7}
		                      & VNLB $W_t = 0$ &         32.11  &         30.88  &         30.18  &         33.30  &         33.19  \\%\cline{2-7}
		                      & VNLB $W_t = 1$ &         33.07  &         31.45  &         30.70  &         34.28  &         34.14  \\%\cline{2-7}
		                      & VNLB $W_t = 2$ & \textbf{33.19} &         31.44  &         30.69  & \textbf{34.34} & \textbf{34.31} \\\hline
%									 
		\multirow{5}{*}{$40$} & BM4D           &         30.42  & \textbf{29.29} &         28.65  &         31.02  & \textbf{31.64} \\%\cline{2-7}
		                      & VNLB $W_t = 0$ &         29.51  &         28.48  &         27.90  &         30.30  &         30.15  \\%\cline{2-7}
		                      & VNLB $W_t = 1$ &         30.62  &         29.19  &         28.61  &         31.31  &         31.11  \\%\cline{2-7}
									 & VNLB $W_t = 2$ & \textbf{30.81} &         29.24  & \textbf{28.66} & \textbf{31.43} &         31.35  \\\hline\hline
	\end{tabular}}

\end{frame}
\restoreframe

% \section{Experimental results}
% 
% %%%%%%%%%%%%%BOB Experiment%%%%%%%%%%%%%%%%%%%%%%%%
% \multipleframe
% 	\begin{frame}{Experimental results: One lid setting}
% 		\begin{center}
% 			\animategraphics[autoplay,loop,height=5cm]{8}{figures_rida/pres/bob_ori/}{002}{032}
% 		\end{center}
% 
% 		\centerline{Fast illumination change. \structure{Original Sequence}}
% 	\end{frame}
% 	
% 	
% 	\begin{frame}{Experimental results: One lid setting}
% 		\begin{center}
% 			\animategraphics[autoplay,loop,height=5cm]{8}{figures_rida/pres/bob_editing/}{002}{032}
% 		\end{center}
% 
% 		\centerline{Fast illumination change: \structure{Editing Domain}}
% 	\end{frame}
% 	
% 	
% 	\begin{frame}{Experimental results: One lid setting}
% 		\begin{center}
% 			\animategraphics[autoplay,loop,height=5cm]{8}{figures_rida/pres/bob_out/}{002}{032}
% 		\end{center}
% 
% 		\centerline{Fast illumination change: \structure{Output}}
% 	\end{frame}
% 
% 	
% \begin{frame}{Experimental results: One lid setting}
% 
% 	\begin{center}
% 		%output
% 		\includegraphics[width=0.195\textwidth]{figures_rida/vid_edit/DSCD_norm_gbc_003}
% 		\includegraphics[width=0.195\textwidth]{figures_rida/vid_edit/DSCD_norm_gbc_013}
% 		\includegraphics[width=0.195\textwidth]{figures_rida/vid_edit/DSCD_norm_gbc_019}
% 		\includegraphics[width=0.195\textwidth]{figures_rida/vid_edit/DSCD_norm_gbc_020}
% 		\includegraphics[width=0.195\textwidth]{figures_rida/vid_edit/DSCD_norm_gbc_029}
% 
% 		%output BC
% 		\includegraphics[width=0.195\textwidth]{figures_rida/vid_edit/DSCD_norm_bc_003}
% 		\includegraphics[width=0.195\textwidth]{figures_rida/vid_edit/DSCD_norm_bc_013}
% 		\includegraphics[width=0.195\textwidth]{figures_rida/vid_edit/DSCD_norm_bc_019}
% 		\includegraphics[width=0.195\textwidth]{figures_rida/vid_edit/DSCD_norm_bc_020}
% 		\includegraphics[width=0.195\textwidth]{figures_rida/vid_edit/DSCD_norm_bc_029}
% 
% 	%	%output BC filtered with mu = 10
% 	%	\includegraphics[width=0.195\textwidth]{figures_rida/vid_edit/DSCD_norm_bcf_m10_003}
% 	%	\includegraphics[width=0.195\textwidth]{figures_rida/vid_edit/DSCD_norm_bcf_m10_013}
% 	%	\includegraphics[width=0.195\textwidth]{figures_rida/vid_edit/DSCD_norm_bcf_m10_019}
% 	%	\includegraphics[width=0.195\textwidth]{figures_rida/vid_edit/DSCD_norm_bcf_m10_020}
% 	%	\includegraphics[width=0.195\textwidth]{figures_rida/vid_edit/DSCD_norm_bcf_m10_029}
% 	\end{center}
% 
% 	\begin{center}
% 	\structure{Norm of the convective derivative.} 
% 	\end{center}
% \end{frame}
% 
% \restoreframe

\begin{frame}{Next steps}
	\begin{itemize}\itemsep=1cm
	  \item Tune parameters - patch size and search region.
	  \item Use coarse motion estimate to displace search region along frames.
	  \item Detect still regions to enforce temporal consistency. 
	  \item Address stair-casing problem.
	  \item Compare against classical video denoising algorithms (e.g. temporal median).
  \end{itemize}
\end{frame}


\begin{frame}{}
	\centering
   \vspace{1.7cm}
   \textcolor{MyOrange}{\textbf{\Huge{Merci.}}}\\
   \vspace{5cm}
\end{frame}



% 
% 

\end{document}

