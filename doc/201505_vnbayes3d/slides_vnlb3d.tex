% \documentclass{beamer}
\documentclass[mathserif, 8pt]{beamer}

\mode<presentation> {
% \usetheme{Madrid}
%\usetheme{default}
\usetheme{Boadilla}
% \usecolortheme{beaver}
% \usecolortheme{rose}
}
% removes navigation symbols
\setbeamertemplate{navigation symbols}{}

\usepackage[english,activeacute]{babel}
\usepackage[english]{layout}
\usepackage{amsfonts,amsmath,amssymb,amsthm}
\usepackage{pgf}
%\usepackage{movie15}
\usepackage{graphicx}
\usepackage{rotating}
\usepackage{color}
\usepackage[utf8]{inputenc}
\usepackage[multidot]{grffile}
\usepackage{ucs}
% \usepackage[T1]{fontenc}
%\usepackage{subfigure}

% from rida's preamble
\usepackage{epsfig,xspace}
\usepackage{setspace}
\usepackage{threeparttable}
\usepackage{subfloat}
\usepackage{pstricks,pst-node}
\usepackage[normal,tight,center]{subfigure}
\usepackage{pifont}
\usepackage{multimedia}
%\usepackage{media9}
%\usepackage{enumitem}
%\usepackage{animate}
\usepackage{caption}
\usepackage{multirow}


% the following is used to bind frames with the same frame number
\usepackage{etoolbox}

\newcounter{multipleslide}

\makeatletter%
\newcommand{\multipleframe}{%
\setcounter{multipleslide}{\value{framenumber}}
\stepcounter{multipleslide}
\patchcmd{\beamer@@tmpl@footline}% <cmd>
	{\insertframenumber}% <search>
	{\themultipleslide}% <replace>
	{}% <success>
	{}% <failure>
}
\newcommand{\restoreframe}{%
\patchcmd{\beamer@@tmpl@footline}% <cmd>
	{\themultipleslide}% <search>
	{\insertframenumber}% <replace>
	{}% <success>
	{}% <failure>
\setcounter{framenumber}{\value{multipleslide}}%
}
\makeatother%




%%%%%%%%%%%%%%%%%%%%%%%%%%%%%%%%%%%%%%%
% Running 'pdflatex --shell-escape...' will automatically recognize
% the eps files in the includegraphics and convert them to pdf.
% No file rename nor change to the document is needed.
%%%%%%%%%%%%%%%%%%%%%%%%%%%%%%%%%%%%%%%
\usepackage{ifpdf}
\ifpdf
\DeclareGraphicsRule{.eps}{pdf}{.pdf}{`epstopdf #1}
\usepackage{epstopdf}
\epstopdfsetup{suffix=-\SourceExt-converted-to}
\pdfcompresslevel=0   %\pdfcompresslevel=9
\pdfcompresslevel0
\fi


% para poder incluir figuras .pdftex del xfig
\DeclareGraphicsRule{.pdftex}{pdf}{.pdftex}{}

\newcommand{\nada}[1]   {}

\newcommand{\best}[1]{\textbf{\textcolor{MyOrange}{#1}}}
\newcommand{\Best}[1]{\textbf{\textcolor{MyOrangeBrighter}{#1}}}

\newcommand{\ma}[1]{\boldsymbol{#1}}
\newcommand{\ie}{\textit{i.e.} }
\newcommand{\eg}{\textit{e.g.} }
\newcommand{\etal}{\textit{et al}. }
\newcommand{\tras}[1]{#1^{\mathrm{T}}}
\newcommand{\herm}[1]{#1^{\mathrm{H}}}
\newcommand{\con}[1]{#1^{\mathrm{*}}}
\newcommand{\E}{\mathbb{E}}
\newcommand{\tech}[1]{\overline{#1}}
\newcommand{\nspace}{\!\!\!\!}
\newcommand{\nmbr}[1]{\oldstylenums{#1}}

\newcommand{\eps}{\varepsilon}
\newcommand{\R}{{\mathbb R}}
\newcommand{\Q}{{\mathbb Q}}
\newcommand{\N}{{\mathbb N}}
\newcommand{\Z}{{\mathbb Z}}
\newcommand{\C}{{\mathcal C}}
\renewcommand{\H}{{\mathcal H}}
\newcommand{\F}{{\mathcal F}}

\DeclareMathOperator*{\argmin}{arg\,min}
\DeclareMathOperator*{\argmax}{arg\,max}
\DeclareMathOperator*{\median}{median}



\definecolor{MyOrange}  {cmyk}{0,0.73,1.0,0}
\definecolor{MyOrangeBrighter}  {cmyk}{0,0.53,7.0,0}
\definecolor{MyGreen}   {cmyk}{1,0,1,0.0}

% Rida's colors and commands
\definecolor{Red}{rgb}{0.9,0.0,0.1}
\definecolor{Brown}{rgb}{0.55,0.27,0.1}
\definecolor{Brownie}{rgb}{0.75,0.27,0.1}
\definecolor{Yellow}{rgb}{1,1,0}
\definecolor{White}{rgb}{1,1,1}
\definecolor{formule}{rgb}{0.75,0.27,0.1}
\definecolor{formule}{rgb}{.7,1,0}
\definecolor{violet}{rgb}{0.7,0,.9}
\definecolor{darkgreen}{rgb}{0,.7,0}

%\newcommand{\reference}[1] {{\scriptsize \color{gray}  #1 }} 
\newcommand{\reference}[1] {{\color{gray}  #1 }} 
\newcommand{\rmk}[1]{{\color{Red} #1}}

%\theoremstyle{plain}\newtheorem{theorem}{Theorem}[chapter]
%\theoremstyle{plain}\newtheorem{proposition}{Proposition}[chapter]
%\theoremstyle{plain}\newtheorem{lemma}{Lemma}[chapter]
%\theoremstyle{definition}\newtheorem{definition}{Definition}[chapter]

% New definition of square root:
% it renames \sqrt as \oldsqrt
\let\oldsqrt\sqrt
% it defines the new \sqrt in terms of the old one
\def\sqrt{\mathpalette\DHLhksqrt}
\def\DHLhksqrt#1#2{%
\setbox0=\hbox{$#1\oldsqrt{#2\,}$}\dimen0=\ht0
\advance\dimen0-0.2\ht0
\setbox2=\hbox{\vrule height\ht0 depth -\dimen0}%
{\box0\lower0.4pt\box2}}


%gets rid of bottom navigation bars
% \setbeamertemplate{footline}[frame number]{}

%gets rid of navigation symbols
\setbeamertemplate{navigation symbols}{}

\title[NLB3D]{$\,$ \\ $\,$ \\ Preliminary results for NL-Bayes 3D}
\author[Pablo Arias]{May 11, 2015}
\institute[CMLA]{}
\date[]{}

%\author[Pablo Arias]{Pablo Arias$^*$, Vicent Caselles$^*$, Gabriele Facciolo$^\dagger$ \\ Rida Sadek$^*$, Guillermo Sapiro$^\ddagger$ \\ $\,$ \\  $\,$ \\ $\,$ \\ $\,$ \\
%\institute[UPF]{* Universitat Pompeu Fabra\\ $\dagger$ ENS Cachan \\ $\ddagger$ Duke University}

% \AtBeginSection[]  % "Beamer, do the following at the start of every section"
% {\begin{frame}<beamer>
% \frametitle{Outline} % make a frame titled "Outline"
% \tableofcontents[currentsection]  % show TOC and highlight current section
% \end{frame}}
%
% \AtBeginSubsection[]  % "Beamer, do the following at the start of every section"
% {\begin{frame}<beamer>
% \frametitle{Outline} % make a frame titled "Outline"
% \tableofcontents[currentsection,currentsubsection]  % show TOC and highlight current section
% \end{frame}}

\graphicspath{{./figures/}}

\begin{document}

\begin{frame}
    \titlepage
\end{frame}

\section{Introduction}

\begin{frame}{Introduction}

	\begin{center}
	Experiments on 5 sequences of the Middlebury optical flow dataset:

	\bigskip

	\emph{Army}, \\
	\emph{DogDance}, \\
	\emph{Evergreen}, \\
	\emph{Mequon}, \\
	\emph{Walking},

	\bigskip

	with white additive Gaussian noise of 

	\bigskip
	
	$\sigma = 10$, \\ 
	$\sigma = 20$, \\ 
	$\sigma = 40$.
	\end{center}

\end{frame}

\begin{frame}{Parameters}
	\begin{center}

	\begin{tabular}{l | c c | c c | c c }
		& \multicolumn{2}{c|}{$\sigma = 10$} 
		& \multicolumn{2}{c|}{$\sigma = 20$} 
		& \multicolumn{2}{c}{$\sigma = 40$} \\
		                            & 1st  & 2nd  & 1st  & 2nd  & 1st  & 2nd \\\hline\hline
		Patch size (spatial)        &  5   &   5  &  5   &   5  &  5   &   5 \\
		Number of patches           & 375  & 375  & 375  & 375  & 375  & 375 \\
		Distance threshold $\tau$   & n/a  & 432  & n/a  & 432  & n/a  & 432 \\
		Spatial search window       & 13   & 37   & 37   & 37   & 37   & 37  \\
		Temporal search range       & 5    & 5    & 5    & 5    & 5    & 5   \\\hline
		Beta                        & 1    & 1.2  & 1    & 1.2  & 1    & 1.2 \\\hline
	\end{tabular}

	\bigskip
	\bigskip
	\bigskip

	We consider varying the temporal size of the patch.\\ All other parameters are kept constant.

	\end{center}
\end{frame}


\multipleframe
\begin{frame}{PSNRs of basic estimates}


	\begin{center}
	{\tiny
	\renewcommand{\tabcolsep}{2mm}
	\renewcommand{\arraystretch}{1.0}
	\begin{tabular}{ c | l |c c c c c}
		% NOTE: These results were obtained with the ACIVS paramter set. These
		% parameters were tuned to the DERF database, but perform a little worse in
		% the Middleburry sequences.
		\hline
		\rule{0pt}{6pt}$\sigma$ & Method           & Army & DogDance & Evergreen & Mequon & Walking  \\\hline
		\multirow{5}{*}{$10$} & V-BM4D-np        & \best{37.48} & \best{35.60} & \best{35.29} & \best{37.89} & \best{38.67} \\%\cline{2-7}
%		                      & VNLB2D $n_t = 0$ &       34.98  &       34.03  &       33.60  &       35.93  &       36.19  \\%\cline{2-7}
%		                      & VNLB2D $n_t = 1$ &       36.49  &       35.02  &       34.56  &       37.65  &       38.06  \\%\cline{2-7}
		                      & VNLB2D $n_t = 5$ &       36.66  &       35.10  &       34.64  &       37.74  &       38.30  \\
		                      & VNLB3D $s_t = 1$ &       todo   &       todo   &       todo   &       todo   &       todo   \\
		                      & VNLB3D $s_t = 2$ &       37.68  &       35.57  &       35.34  &       37.72  &       38.54  \\
		                      & VNLB3D $s_t = 3$ &       37.60  &       35.58  &       35.33  &       37.37  &       38.18  \\
		                      & VNLB3D $s_t = 4$ &       37.20  &       35.38  &       35.16  &       36.99  &       37.67  \\\hline
%
		\multirow{5}{*}{$20$} & V-BM4D-np        & \best{32.24} & \best{31.07} & \best{30.48} &       32.89  & \best{33.54} \\%\cline{2-7}
%		                      & VNLB2D $n_t = 0$ &       31.45  &       30.75  &       30.28  &       32.29  &       32.43  \\%\cline{2-7}
%		                      & VNLB2D $n_t = 1$ &       33.13  &       31.95  &       31.42  &       34.15  &       34.35  \\%\cline{2-7}
		                      & VNLB2D $n_t = 5$ &       33.38  &       32.09  &       31.56  &       34.39  &       34.64  \\
		                      & VNLB3D $s_t = 1$ &       todo   &       todo   &       todo   &       todo   &       todo   \\
		                      & VNLB3D $s_t = 2$ &       34.43  &       32.70  &       32.29  &       34.73  &       35.30  \\
		                      & VNLB3D $s_t = 3$ &       34.42  &       32.71  &       32.31  &       34.50  &       35.14  \\
		                      & VNLB3D $s_t = 4$ &       34.09  &       32.53  &       32.15  &       34.16  &       34.79  \\\hline
%
		\multirow{5}{*}{$40$} & V-BM4D-np        & \best{29.53} & \best{28.59} &       27.93  &       29.94  & \best{30.55} \\%\cline{2-7}
%		                      & VNLB2D $n_t = 0$ &       26.91  &       26.53  &       26.19  &       27.45  &       27.53  \\%\cline{2-7}
%		                      & VNLB2D $n_t = 1$ &       29.06  &       28.25  &       27.84  &       29.72  &       29.84  \\%\cline{2-7}
		                      & VNLB2D $n_t = 5$ &       29.44  &       28.52  &       28.10  &       30.09  &       30.24  \\
		                      & VNLB3D $s_t = 1$ &       todo   &       todo   &       todo   &       todo   &       todo   \\
		                      & VNLB3D $s_t = 2$ &       30.49  &       29.26  &       28.88  &       30.66  &       31.07  \\
		                      & VNLB3D $s_t = 3$ &       30.44  &       29.24  &       28.86  &       30.46  &       30.97  \\
		                      & VNLB3D $s_t = 4$ &       30.06  &       29.00  &       28.61  &       30.09  &       30.65  \\\hline
		\end{tabular}}

	\bigskip

	PSNRs obtained for the five color sequences from the Middlebury dataset.

	\end{center}

%	\begin{center}
%	\begin{tabular}{| c | c |c c c c c|}
%		\hline \hline
%		$\sigma$  & Method \textbackslash Video & Army & DogDance & Evergreen & Mequon & Walking \\\hline\hline
%		\multirow{5}{*}{$10$} & BM4D           & \best{37.48} & \best{35.60} & \best{35.29} & \best{37.89} & \best{38.67} \\%\cline{2-7}
%		                      & VNLB $W_t = 0$ &       14.30  &       18.98  &       15.87  &       16.83  &       16.28  \\%\cline{2-7}
%		                      & VNLB $W_t = 1$ &       36.21  &       34.86  &       34.50  &       37.16  &       37.54  \\%\cline{2-7}
%		                      & VNLB $W_t = 2$ &       36.77  &       35.22  &       34.82  &       37.69  &       38.20  \\\hline
%%
%		\multirow{5}{*}{$25$} & BM4D           & \best{32.24} & \best{31.07} & \best{30.48} &       32.89  & \best{33.54} \\%\cline{2-7}
%		                      & VNLB $W_t = 0$ &       30.13  &       29.51  &       29.07  &       30.91  &       30.00  \\%\cline{2-7}
%		                      & VNLB $W_t = 1$ &       31.92  &       30.85  &       30.33  &       32.86  &       33.00  \\%\cline{2-7}
%		                      & VNLB $W_t = 2$ & \best{32.21} & \best{31.02} & \best{30.50} & \best{33.14} &       33.31  \\\hline
%%
%		\multirow{5}{*}{$40$} & BM4D           & \best{29.53} & \best{28.59} &       27.93  &       29.94  & \best{30.55} \\%\cline{2-7}
%		                      & VNLB $W_t = 0$ &       27.03  &       26.62  &       26.27  &       27.59  &       27.65  \\%\cline{2-7}
%		                      & VNLB $W_t = 1$ &       29.13  &       28.30  &       27.88  &       29.79  &       29.91  \\%\cline{2-7}
%		                      & VNLB $W_t = 2$ & \best{29.50} & \best{28.55} & \best{28.13} & \best{30.15} &       30.29  \\\hline\hline
%	\end{tabular}
%
%	\bigskip
%	Results of VNLB corresponding to patch sizes of $5$ in both steps.\\The rest of parameters are the default ones.
%	\end{center}

\end{frame}

\begin{frame}{PSNRs of final estimates}


	\begin{center}
	{\tiny
	\renewcommand{\tabcolsep}{2mm}
	\renewcommand{\arraystretch}{1.0}
	\begin{tabular}{ c | l |c c c c c}
		% NOTE: These results were obtained with the ACIVS paramter set. These
		% parameters were tuned to the DERF database, but perform a little worse in
		% the Middleburry sequences.
		\hline
		\rule{0pt}{6pt}$\sigma$ & Method        & Army & DogDance & Evergreen & Mequon & Walking  \\\hline
		\multirow{5}{*}{$10$} & V-BM4D-np        &       37.77  &       35.70  &       35.40  &       38.09  &       38.85  \\
%		                      & VNLB2D $n_t = 0$ &       36.90  &       35.42  &       34.84  &       38.43  &       38.86  \\
%		                      & VNLB2D $n_t = 3$ &       37.65  &       35.87  &       35.38  &       39.09  &       39.69  \\
		                      & VNLB2D $n_t = 5$ & \best{37.88} & \best{36.01} & \best{35.53} & \Best{39.20} & \best{39.74} \\
		                      & VNLB3D $s_t = 1$ &       todo   &       todo   &       todo   &       todo   &       todo   \\
		                      & VNLB3D $s_t = 2$ &       38.69  &       36.26  &       35.97  & \best{39.02} & \Best{39.95} \\
		                      & VNLB3D $s_t = 3$ &       39.06  & \Best{36.48} & \Best{36.20} & \best{38.96} & \Best{39.97} \\
		                      & VNLB3D $s_t = 4$ & \Best{39.17} & \Best{36.55} & \Best{36.29} &       38.86  & \Best{39.90} \\\hline
%
		\multirow{5}{*}{$20$} & V-BM4D-np        &       32.91  &       31.50  &       30.94  &       33.60  &       34.27  \\
%		                      & VNLB2D $n_t = 0$ &       33.77  &       32.44  &       31.68  &       35.61  &       35.60  \\
%		                      & VNLB2D $n_t = 3$ &       34.45  &       32.90  &       32.15  &       35.89  &       36.12  \\
		                      & VNLB2D $n_t = 5$ & \best{34.59} & \best{33.02} & \best{32.27} & \best{35.90} & \best{36.12} \\
		                      & VNLB3D $s_t = 1$ &       todo   &       todo   &       todo   &       todo   &       todo   \\
		                      & VNLB3D $s_t = 2$ &       35.55  &       33.37  &       32.78  & \Best{36.19} &       36.95  \\
		                      & VNLB3D $s_t = 3$ &       35.94  & \Best{33.58} &       33.06  & \Best{36.13} & \Best{37.07} \\
		                      & VNLB3D $s_t = 4$ & \Best{36.08} & \Best{33.65} & \Best{33.20} &       36.04  & \Best{37.05} \\\hline
%
		\multirow{5}{*}{$40$} & V-BM4D-np        &       30.42  &       29.29  &       28.65  &       31.02  &       31.64  \\
%		                      & VNLB2D $n_t = 0$ &       30.55  &       29.41  &       28.66  &       31.78  &       31.86  \\
%		                      & VNLB2D $n_t = 3$ & \best{31.02} & \best{29.73} & \best{29.03} & \best{31.88} & \best{31.92} \\
		                      & VNLB2D $n_t = 5$ & \best{31.06} & \best{29.74} & \best{29.08} & \best{31.70} & \best{31.74} \\
		                      & VNLB3D $s_t = 1$ &       todo   &       todo   &       todo   &       todo   &       todo   \\
		                      & VNLB3D $s_t = 2$ &       32.60  &       30.63  &       29.91  & \Best{33.00} &       33.63  \\
		                      & VNLB3D $s_t = 3$ &       33.03  & \Best{30.87} &       30.20  & \Best{33.05} & \Best{33.89} \\
		                      & VNLB3D $s_t = 4$ & \Best{33.17} & \Best{30.95} & \Best{30.34} & \Best{33.00} & \Best{33.95} \\\hline
%
		\end{tabular}}

	\bigskip
	
	PSNRs obtained for the five color sequences from the Middlebury dataset.

	\end{center}





%	\begin{center}
%	\begin{tabular}{| c | c |c c c c c|}
%		% NOTE: These results were obtained with a set of experiments run in
%		order to tune the parametrs.
%		\hline \hline
%		$\sigma$  & Method \textbackslash Video & Army & DogDance & Evergreen & Mequon & Walking \\\hline\hline
%		\multirow{5}{*}{$10$} & BM4D           & \best{37.77} &       35.70  & \best{35.40} &       38.09  &       38.85  \\%\cline{2-7}
%		                      & VNLB $W_t = 0$ &       34.46  &       33.61  &       33.05  &       35.65  &       35.90  \\%\cline{2-7}
%		                      & VNLB $W_t = 1$ &       37.65  & \best{35.87} & \best{35.42} & \best{39.10} & \best{39.70} \\%\cline{2-7}
%		                      & VNLB $W_t = 2$ & \best{37.75} & \best{35.92} & \best{35.47} & \best{39.06} & \best{39.79} \\\hline
%%
%		\multirow{5}{*}{$25$} & BM4D           &       32.91  &       31.50  &       30.94  &       33.60  &       34.27  \\%\cline{2-7}
%		                      & VNLB $W_t = 0$ &       32.33  &       31.17  &       30.39  &       33.69  &       33.74  \\%\cline{2-7}
%		                      & VNLB $W_t = 1$ &       33.48  & \best{31.98} & \best{31.22} & \best{35.10} &       35.25  \\%\cline{2-7}
%		                      & VNLB $W_t = 2$ & \best{33.58} & \best{31.98} & \best{31.22} & \best{35.08} & \best{35.38} \\\hline
%%
%		\multirow{5}{*}{$40$} & BM4D           &       30.42  &       29.29  &       28.65  &       31.02  &       31.64  \\%\cline{2-7}
%		                      & VNLB $W_t = 0$ &       29.94  &       28.95  &       28.24  &       30.98  &       30.95  \\%\cline{2-7}
%		                      & VNLB $W_t = 1$ &       31.36  &       29.96  & \best{29.19} & \best{32.70} &       32.71  \\%\cline{2-7}
%		                      & VNLB $W_t = 2$ & \best{31.55} & \best{30.01} & \best{29.24} & \best{32.75} & \best{32.94} \\\hline\hline
%	\end{tabular}
%
%	\bigskip
%	Results of VNLB corresponding to patch sizes of $5$ in both steps.\\The rest of parameters are the default ones.
%	\end{center}

\end{frame}
\restoreframe

\multipleframe
\begin{frame}{PSNRs of final estimates}
	\begin{center}
		{\tiny
		\renewcommand{\tabcolsep}{3mm}
		\renewcommand{\arraystretch}{1.0}
		\begin{tabular}{ c | l |c c c c}
			\hline
			\rule{0pt}{6pt}$\sigma$ & Method             & Tennis       & Coastguard   & Foreman      & Bus          \\\hline
			\multirow{5}{*}{$10$} & V-BM4D [Maggioni'12] & \best{36.42} & \best{37.27} &       37.92  &       36.23  \\
			                      & V-BM4D-mp            &       35.90  &       36.30  &       37.21  &       35.38  \\
			                      & V-BM4D-np            &       35.56  &       36.20  &       36.90  &       35.09  \\
			                      & V-BM3D               &       36.04  &       36.82  &       37.52  &       34.96  \\
%			                      & VNLB $n_t = 0$       &       34.43  &       36.70  &       37.21  &       35.58  \\
%			                      & VNLB $n_t = 3$       &       35.04  &       37.19  &       37.88  &       36.19  \\
			                      & VNLB $n_t = 5$       &       35.28  & \best{37.30} & \best{38.11} & \best{36.37} \\
			                      & VNLB3D $s_t = 1$     &       todo   &       todo   &       todo   &       todo   \\
			                      & VNLB3D $s_t = 2$     &       35.95  &       37.81  &       38.46  &       36.31  \\
			                      & VNLB3D $s_t = 3$     &       36.34  &       38.16  &       38.74  &       36.50  \\
			                      & VNLB3D $s_t = 4$     & \Best{36.54} & \Best{38.31} & \Best{38.86} & \Best{36.60} \\\hline
%
			\multirow{5}{*}{$20$} & V-BM4D [Maggioni'12] & \Best{32.88} & \best{33.61} & \best{34.62} & \best{32.27} \\
			                      & V-BM4D-mp            &       31.98  &       32.44  &       33.70  &       31.34  \\
			                      & V-BM4D-np            &       31.67  &       32.24  &       33.34  &       30.95  \\
			                      & V-BM3D               &       32.54  &       33.39  &       34.49  &       31.03  \\
%			                      & VNLB $n_t = 0$       &       30.65  &       32.70  &       33.64  &       31.39  \\
%			                      & VNLB $n_t = 3$       &       31.26  &       33.35  &       34.25  &       32.11  \\
			                      & VNLB $n_t = 5$       &       31.49  & \best{33.55} &       34.45  & \best{32.35} \\
			                      & VNLB3D $s_t = 1$     &       todo   &       todo   &       todo   &       todo   \\
			                      & VNLB3D $s_t = 2$     &       32.19  &       34.08  &       34.98  &       32.25  \\
			                      & VNLB3D $s_t = 3$     &       32.69  &       34.34  &       35.33  &       32.49  \\
			                      & VNLB3D $s_t = 4$     & \Best{32.96} & \Best{34.48} & \Best{35.48} & \Best{32.66} \\\hline
%
			\multirow{5}{*}{$40$} & V-BM4D [Maggioni'12] & \best{29.52} & \best{30.00} & \best{31.30} &       28.32  \\
			                      & V-BM4D-mp            &       28.14  &       28.73  &       30.09  &       27.44  \\
			                      & V-BM4D-np            &       27.97  &       28.43  &       29.69  &       27.02  \\
			                      & V-BM3D               &       29.20  & \best{29.99} &       31.17  &       27.34  \\
%			                      & VNLB $n_t = 0$       &       27.43  &       28.88  &       30.15  &       27.60  \\
%			                      & VNLB $n_t = 3$       &       27.97  &       29.60  &       30.72  &       28.23  \\
			                      & VNLB $n_t = 5$       &       28.11  &       29.78  &       30.82  & \best{28.40} \\
			                      & VNLB3D $s_t = 1$     &       todo   &       todo   &       todo   &       todo   \\
			                      & VNLB3D $s_t = 2$     &       28.89  &       30.63  &       31.67  &       28.48  \\
			                      & VNLB3D $s_t = 3$     &       29.37  &       30.95  &       32.07  &       28.72  \\
			                      & VNLB3D $s_t = 4$     & \Best{29.64} & \Best{31.09} & \Best{32.26} & \Best{28.88} \\\hline
		\end{tabular}}
% Command to print rounded psnrs
% for i in $(cat bus_s40_pt*/measures | grep PSNR_final | sed "s/^-PSNR_final\ =\ "//); do  echo "scale=2;(((10^2)*$i)+0.5)/(10^2)" | bc; done

		\bigskip

		PSNRs obtained for the four classic color test sequences.
	\end{center}
\end{frame}
\restoreframe

\begin{frame}{Computation time}

	Computations performed in \texttt{boucantrin} server, using 8 CPU cores.\\
	For CIF (352x288) RGB videos, with $\sigma = 40$: 

	\bigskip

	\begin{center}
	\begin{tabular}{r | c c c c}
		$p_t$ & 1 & 2 & 3 & 4 \\
		$s/\textnormal{frame}$ & x & 7 & 15 & 30 \\
	\end{tabular}
	\end{center}

	\vspace{2cm}

	For example, with $p_t = 4$, it takes 2.5 hours ($\times 8$ cores) for 300 CIF frames!

\end{frame}
% 
% 

\end{document}

