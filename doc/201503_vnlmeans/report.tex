\documentclass[a4paper,10pt]{article}

\usepackage[english,activeacute]{babel}
\usepackage[english]{layout}
\usepackage{amsfonts,amsmath,amssymb,amsthm}
\usepackage[dvips]{graphicx}
\usepackage{color}
\usepackage{subfig}
\usepackage{colonequals}
\usepackage[latin1]{inputenc}
\usepackage{url}
\usepackage{epsfig}
%\usepackage{dsfont}

\usepackage{algorithm}
\usepackage{algorithmic}
\usepackage{setspace}
\usepackage{afterpage}
%\usepackage{showkeys}
% \usepackage{mathptmx}      % use Times fonts if available on your TeX system
% \usepackage{latexsym}

\usepackage[left=1.2in,right=1.2in]{geometry}

% \graphicspath{{../iccv09/}}

% NOTE: COMMENT !!
\newcommand{\mathds}[1]{\mathbb{#1}}

\newcommand{\ma}[1]{\boldsymbol{#1}}
\newcommand{\tras}[1]{#1^{\mathrm{T}}}
\newcommand{\herm}[1]{#1^{\mathrm{H}}}
\newcommand{\con}[1]{#1^{\mathrm{*}}}
\newcommand{\E}{\mathds{E}}
\newcommand{\tech}[1]{\overline{#1}}
\newcommand{\nspace}{\!\!\!\!}
\newcommand{\nmbr}[1]{\oldstylenums{#1}}

\newcommand{\eg}{\emph{e.g}. } \newcommand{\Eg}{\emph{E.g}. }
\newcommand{\ie}{\emph{i.e}. } \newcommand{\Ie}{\emph{I.e}. }
\newcommand{\cf}{\emph{c.f}. } \newcommand{\Cf}{\emph{C.f}. }
\newcommand{\etc}{\emph{etc}. } \newcommand{\vs}{\emph{vs}. }
\newcommand{\wrt}{w.r.t\onedot } \newcommand{\dof}{d.o.f. }
\newcommand{\etal}{\emph{et al}. }

\newcommand{\R}{\mathds{R}}
\newcommand{\sign}{\mathrm{sign}}
\newcommand{\eps}{\varepsilon}
\newcommand{\To}{\longrightarrow}
% \newcommand{\II}{1{\hskip -2.5 pt}\hbox{I}}

% c++ code
\usepackage{listings}
\usepackage{xcolor}
\usepackage{textcomp}
\definecolor{listinggray}{gray}{0.9}
\definecolor{lbcolor}{rgb}{0.9,0.9,0.9}
\lstset{
	backgroundcolor=\color{lbcolor},
	tabsize=4,    
%  rulecolor=,
	language=Matlab,
	basicstyle=\scriptsize\ttfamily,
	upquote=true,
	aboveskip={1.5\baselineskip},
	columns=fixed,
	showstringspaces=false,
	extendedchars=false,
	breaklines=true,
	prebreak = \raisebox{0ex}[0ex][0ex]{\ensuremath{\hookleftarrow}},
	frame=single,
	numbers=left,
	showtabs=false,
	showspaces=false,
	showstringspaces=false,
	identifierstyle=\ttfamily,
	keywordstyle=\color[rgb]{0,0,1}\ttfamily,
	commentstyle=\color[rgb]{0,0.5,0}\ttfamily,
	stringstyle=\color[rgb]{0.627,0.126,0.941}\ttfamily,
	numberstyle=\color[rgb]{0.5, 0.5, 0.5}\ttfamily,
%  \lstdefinestyle{C++}{language=C++,style=numbers}’.
}
%\lstset{
%	backgroundcolor=\color{lbcolor},
%	tabsize=4,
%	language=C++,
%	captionpos=b,
%	tabsize=3,
%	frame=lines,
%	numbers=left,
%	numberstyle=\tiny,
%	numbersep=5pt,
%	breaklines=true,
%	showstringspaces=false,
%	basicstyle=\footnotesize,
%	identifierstyle=\color{magenta},
%	keywordstyle=\color[rgb]{0,0,1},
%	commentstyle=\color{green},
%	stringstyle=\color{red}
%}
	
	
	


\DeclareMathOperator*{\argmin}{arg\,min}
\DeclareMathOperator*{\argmax}{arg\,max}

\newcommand{\nota}[1]{\textcolor{blue}{\textbf{#1}}}
\newcommand{\suggest}[1]{\textcolor{yellow}{#1}}
\newcommand{\add}[1]{\textcolor{green}{#1}}
\newcommand{\remove}[1]{\textcolor{red}{#1}}
\newcommand{\dosdv}[1] {#1}
\newcommand{\uncite}[1] {}
% \newcommand{\tachar}[1]{
% \setbox4=\hbox{\ } \setbox3=\hbox{#1} \hbox{#1} \kern -\wd3 \kern
% -\wd4 \raise 0.3\ht3 \hbox{ \vrule width \wd3 height 0.5pt} }

% \theoremstyle{plain}\newtheorem{theorem}{Theorem}[chapter]
% \theoremstyle{plain}\newtheorem{proposition}{Proposition}[chapter]
% \theoremstyle{plain}\newtheorem{lemma}{Lemma}[chapter]
% \theoremstyle{definition}\newtheorem{definition}{Definition}[chapter]

% New definition of square root:
% it renames \sqrt as \oldsqrt
\let\oldsqrt\sqrt
% it defines the new \sqrt in terms of the old one
\def\sqrt{\mathpalette\DHLhksqrt}
\def\DHLhksqrt#1#2{%
\setbox0=\hbox{$#1\oldsqrt{#2\,}$}\dimen0=\ht0
\advance\dimen0-0.2\ht0
\setbox2=\hbox{\vrule height\ht0 depth -\dimen0}%
{\box0\lower0.4pt\box2}}

\title{Video denoising with non-local means}
\author{}
\date{}

\begin{document}

\maketitle
\abstract{Brief description of an extension of the non-local
means algorithm to video denoising.}

\section*{Overview}

This implementation of non-local means comes from modifying the non-local Bayes
algorithm. The only difference is that groups patches are denoised via averages
weighted by similarity, instead of the empirical Bayes estimate of non-local Bayes.

The code in Listing \ref{src:main-overview} shows an overview of the the
algorithm. The first step computes an initial basic estimate which is used as an
oracle for the second step. In the second step, the basic estimate is used
to compute patch distances and similarity weights. We now provide some details
about both steps:

\section*{Step 1: basic estimate}

\paragraph{Patch distances:} The video is converted to YUV colorspace, and only
the Y component is used for the computation of distances and similarity
weights.

\paragraph{Search for similar patches:} We search for similar patches of size
$s_1\times s_1$ in a spatio-temporal window of size $W_{x,1}\times W_{x,1}\times
W_{t,1}$ centered around the current pixel $x = (x_0, x_1, t)$. From all
patches in the search window, we keep only
the $N_{\text{sim},1}$ most similar. This set of patches is the output of the
patch search stage. We denoted it by $\mathcal G_x$, and call it \verb+group+
in the pseudo-code.

\paragraph{Non-local means estimate:} The group of $N_{\text{sim},1}$ patches
is denoised via similarity weighted averages. Consider a noisy patch $\hat p_y\in \mathcal
G_x$. Then its non-local means estimate is given by
\[p_y^{\text{NLM},1} = \frac1{C(\hat p_y; \mathcal G_x)}\sum_{\hat p_z\in\mathcal G_x} w(\hat p_y, \hat p_z)\hat p_z,\]
where $w(\hat p_y, \hat p_z)$ is the similarity weight between patches $\hat
p_y$ and $\hat p_z$ and $C(\hat p_y; \mathcal G_x)$ is a normalization
constant, and the similarity is computed following BUADESIPOL:
\[w(\hat p_y, \hat p_z) = \exp\left(-\frac1{h_1^2}\max(\|\hat p_y - \hat p_z\|_Y^2, 2\sigma^2/3)\right),\]
where $\|\,\cdot\,\|_Y$ is the Euclidean norm of the Y component. The factor
$1/3$ in the above formula is because we are working with the $Y$
component.

\paragraph{Aggregation:} The estimated pathes of $\mathcal G_x$
are aggregated into their original positions. The result of this aggregation is
the output of the first step: the \emph{basic estimate}. The positions of these
patches are labeled as processed. Optionally to reduce computations, we can label
their 4 spatial neighbors as being also processed, as in NL-Bayes.

\section*{Step 2: final estimate}

\paragraph{Search for similar patches:} Patches are of size
$s_2\times s_2$, and the search is performed in a spatio-temporal window of
size $W_{x,2}\times W_{x,2}\times W_{t,2}$ centered around the current pixel $x
= (x_0, x_1, t)$. In the second stage, the distance between patches is computed
using the RGB components of the basic estimate. Two sets of patches are built,
taking patches from the basic estimate in $\mathcal G^b_x$ and their noisy versions
in $\mathcal G_x^n$.

The number of similar patches kept is adaptive. Let $d_{N_{\text{sim},2}}$
be the distance between the reference patch and its $N_{\text{sim},2}$ most similar patch.
All patches with a distance
to the reference patch smaller than $\max(\tau_2, d_{N_{\text{sim},2}})$ are kept.
Thus, $N_{\text{sim},2}$ is a lower bound on the number of similar patches.

\paragraph{Non-local means estimate:} As in step 1, the group $\mathcal G_x^n$ of patches
is denoised via similarity weighted averages. The similarity weight is computed
from the corresponding patches in $\mathcal G_x^b$. Thus, for a noisy patch
$\hat p_y\in \mathcal
G_x^n$, we have:
\[p_y^{\text{NLM},2} = \frac1{C(p^b_y; \mathcal G^b_x)}\sum_{\hat p_z\in\mathcal G^n_x} w(p^b_y, p^b_z)\hat p_z.\]
The similarity $w(\hat p_y, \hat p_z)$ is computed as in step 1, but using the
RGB components:
\[w(p^b_y, p^b_z) = \exp\left(-\frac1{h_2^2}\max(\|p^b_y - p^b_z\|_{\text{RGB}}^2, 2\sigma^2)\right).\]

\paragraph{Aggregation:} As in step 1.


\begin{lstlisting}[%
	float=h,
	caption={Overview of main function},
	label=src:main-overview,
	xleftmargin=.9cm,
	xrightmargin=.9cm]
function final_vid = videoNLMeans(noisy_vid, params1, params2)

%%%%%%%%
% step 1
%%%%%%%%

% convert to YUV colorspace
noisy_vid = convertRGBtoYUV(noisy_vid);

% label all pixels as non-processed
processed = Video(width, height, nframes, false);

% initialize basic estimate as zero
basic_vid = Video(width, height, nframes, channels, 0);

% process video
for x = 1:noisy_vid.nPixels,
	if processed(x) == false,
		% search for similar patches around the reference one
		group = getSimilarPatchesStep1(noisy_vid, x, params1);

		% denoise patches in group via similarity-weighted averages
		group = nlMeansEstimateStep1(group, params1);

		% aggregate group over basic estimate. update processed pixels
		[basic_vid, processed] = aggregateStep1(basic_vid, processed, group, params1);
	end
end

% convert back to RGB colorspace
noisy_vid = convertYUVtoRGB(noisy_vid);
basic_vid = convertYUVtoRGB(basic_vid);

%%%%%%%%
% step 2
%%%%%%%%

% label all pixels as non-processed
processed = Video(width, height, nframes, false);

% initialize final estimate as zero
final_vid = Video(width, height, nframes, channels, 0);

% process video
for x = 1:noisy_vid.nPixels,
	if processed(x) == false,
		% search for similar patches around the reference one
		[group_noisy, group_basic] = getSimilarPatchesStep2(noisy_vid, basic_vid, x, params2);

		% denoise patches in group via similarity-weighted averages
		group_basic = nlMeansEstimateStep2(group_noisy, group_basic, params2);

		% aggregate group estimate, and update mask of processed pixels
		[final_vid, processed] = aggregateStep2(final_vid, processed, group, params2);
	end
end
\end{lstlisting}


\section*{Parameters}

Parameters used in the patch search: $W_{x,i}, W_{t,i}, N_{\text{sim},i}$ and
$\tau_2$, are computed as a function of $\sigma$, following the empirical rules
provided in LEBRUNIPOL. 

Parameters $s_1$ and $h_1$ are computed as a function of $\sigma/\sqrt{3}$ following 
the empirical rules provided in BUADESIPOL for color images. In principle,
the parameters for grayscale images of BUADESIPOL should be better since weights are
computed based on the Y component, but performed worse on the Middleburry
sequences.

Parameters $s_2$ and $h_2$ were set by trial and error.


\end{document}
